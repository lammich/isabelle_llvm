\documentclass[fleqn]{beamer}


\mode<presentation>
{
  \usetheme{default}
%   \useinnertheme[shadow=true]{rounded}

  \useinnertheme{circles}
  
%    \useoutertheme{infolines}
  % or ...
  \setbeamersize{text margin left=1em,text margin right=1em}

%   \setbeamercovered{transparent}
  % or whatever (possibly just delete it)

  \beamertemplatenavigationsymbolsempty
  
% Display frame numbers in footline
  \setbeamertemplate{footline}[frame number]
}

\usepackage{etex}
\usepackage[utf8]{inputenc}
\usepackage[T1]{fontenc}
\usepackage[english]{babel}
\usepackage{amsmath}
% \usepackage{amsthm}
% \usepackage{stmaryrd}
\usepackage{lmodern}
\usepackage{mathpartir}

\usepackage{centernot}

\usepackage{colortbl}
\usepackage{multirow}

\usepackage[purexy]{qsymbols}
\usepackage{graphicx}

\usepackage{cprotect}
\usepackage{listings}
\usepackage{lstautogobble}
\usepackage{relsize}

\usepackage[noend]{algpseudocode}

\usepackage{ifthen}
\usepackage{pgf}
\usepackage{tikz}
\usetikzlibrary{scopes}
\usetikzlibrary{decorations}
  \usetikzlibrary{decorations.pathmorphing}
\usetikzlibrary{arrows}
\usetikzlibrary{automata}
\usetikzlibrary{positioning}
\usetikzlibrary{chains}
\usetikzlibrary{shapes.geometric}
\usetikzlibrary{shapes.callouts}
\usetikzlibrary{shapes.misc}
\usetikzlibrary{shapes.multipart}
\usetikzlibrary{fit}
\usetikzlibrary{calc}

\usepackage{pgfplots}

% \tikzstyle{stack}=[inner sep=0pt,minimum size=2mm]
% \tikzstyle{ssline}=[->,snake=snake,segment amplitude=.2mm,segment length=3mm,line after snake=1mm]
% \tikzstyle{fgnode}=[circle,draw,inner sep=0pt,minimum size=2mm]

\tikzset{
  -|-/.style={
    to path={
      (\tikztostart) -| ($(\tikztostart)!#1!(\tikztotarget)$) |- (\tikztotarget)
      \tikztonodes
    }
  },
  -|-/.default=0.5,
  |-|/.style={
    to path={
      (\tikztostart) |- ($(\tikztostart)!#1!(\tikztotarget)$) -| (\tikztotarget)
      \tikztonodes
    }
  },
  |-|/.default=0.5,
}


\tikzset{
  invisible/.style={opacity=0},
  visible on/.style={alt=#1{}{invisible}},
  alt/.code args={<#1>#2#3}{%
    \alt<#1>{\pgfkeysalso{#2}}{\pgfkeysalso{#3}} % \pgfkeysalso doesn't change the path
  },
}



% The goal is to translate
%    \overlaynode<red,blue>{hallo};
% into
%    \node[red]{hallo};
%    \node[blue]{hallo};
\makeatletter
\def\overlaynode<#1>#2;{
        \gdef\stacknodecommonpart{#2}
        \pgfkeys{/typeset node/.list={#1}}
        % we are lazy
        % pgfkeys will translate
        %    \pgfkeys{/typeset node/.list={red,blue}}
        % into
        %    \pgfkeys{/typeset node=red}
        %    \pgfkeys{/typeset node=blue}
}
\pgfkeys{
    /typeset node/.code={
        \edef\pgf@marshal{\noexpand\node[#1]\stacknodecommonpart;}
        \pgf@marshal
    }
}
% \tikz{
%     \overlaynode<red,{blue,xshift=1}>{{Hello}}; % notice the nested {{}}
% %   \overlaynode<red,{blue,xshift=1}>[]{Hello}; % workaround
%     \overlaynode<red,{blue,rotate=5}>at(1,0)[draw]{from};
%     \overlaynode<red,{blue,scale=1.1}>[circle]at(2,0)[draw]{the};
%     \overlaynode<red,{blue,opacity=.5}>[fill=yellow!50]at(3,0){other};
%     \overlaynode<{red,rectangle callout,fill},{blue,ellipse callout,draw}>at(4,0){side};
% }
% \tikz{
%     \overlaynode<
%             {fill=red,callout absolute pointer=(45:2)},
%             {fill=yellow,callout absolute pointer=(135:2),callout pointer shorten=1cm},
%             {fill=green,callout relative pointer=(-135:2),callout pointer width=.5cm},
%             {fill=blue,callout relative pointer=(-45:2),callout pointer shorten=1cm,callout pointer width=.5cm}
%         >
%         [rectangle callout,text=white]
%         at(0,0){can you}
%     ;
% }

\def\overlaynodedrawfill{\pgfutil@ifnextchar[{\overlaynodedrawfill@opt}{\overlaynodedrawfill@opt[]}}
\def\overlaynodedrawfill@opt[#1]<#2>#3;{
    \begin{scope}[transparency group,draw=black,fill=white,line cap=round,line join=round,#1]
        \pgfmathsetmacro\pgflinewidthdouble{2\pgflinewidth}
        \overlaynode<#2>[draw=pgfstrokecolor,line width=\pgflinewidthdouble]#3;
        \overlaynode<#2>[fill=pgffillcolor]#3;
    \end{scope}
}

\makeatother


% \tikz{
%     \overlaynodedrawfill[draw=magenta,fill=cyan,opacity=.5]<
%             {callout absolute pointer=(45:2)},
%             {callout absolute pointer=(135:2),callout pointer shorten=1cm},
%             {callout relative pointer=(-135:2),callout pointer width=.5cm},
%             {callout relative pointer=(-45:2),callout pointer shorten=1cm,callout pointer width=.5cm}
%         >
%         [rectangle callout]
%         at(0,0){hear me}
%     ;
% }
%
% \tikz{
%     \overlaynodedrawfill<
%             {cloud callout,callout absolute pointer=(90:2),inner sep=-20},
%             {rectangle callout,callout absolute pointer=(210:2),minimum height=30},
%             {ellipse callout,callout absolute  pointer=(-30:2)}
%         >
%         at(0,0){nice to meet you}
%     ;
% }
%
% \tikz{
%     \overlaynodedrawfill<
%             {starburst},
%             {cloud,inner sep=-10}
%         >
%         at(0,0){where you been}
%     ;
% }
%
% \usetikzlibrary{shapes.arrows}
% \tikz{
%     \overlaynodedrawfill[arrow box,text opacity=0,minimum height=50,minimum width=50,inner xsep=-10]<
%             {rotate=22.5},
%             {rotate=45},
%             {rotate=67.5},
%             {text opacity=1}
%         >
%         at(0,0){I can}
%     ;
% }
%








\usepackage{packages/isabelle}
\usepackage{packages/isabelletags}
\usepackage{packages/isabellesym}
\usepackage{packages/comment}

% \isabellestyle{it}

\def\isachardoublequote{}%
\def\isachardoublequoteopen{}%
\def\isachardoublequoteclose{}%

\newcommand{\isainnerkeyword}[1]{{\textbf{#1}}}
\newcommand{\isasymexistsA}{\isamath{\exists_{\textsc A}\,}}


\def\isadelimproof{}
\def\endisadelimproof{}
\def\isatagproof{}
\def\endisatagproof{}
\def\isafoldproof{}
\def\isadelimproof{}
\def\endisadelimproof{}

\def\isastylescript{\sl}%


% Not really meant for highlighting isabelle source, but for easily writing latex that looks like
% isabelle
% 
% keyword level 1 - isabelle outer syntax 
% keyword level 2 - isabelle inner syntax programming constructs (if, let, etc)
% keyword level 3 - standard constants (length, mod, etc)
% keyword level 4 - isabelle proof methods


% \newcommand{\lsem}{\ensuremath{\mathopen{[\![}}}
% \newcommand{\rsem}{\ensuremath{\mathclose{]\!]}}}

\lstdefinelanguage{isabelle}{
  morekeywords={theorem,theorems,corollary,lemma,lemmas,locale,begin,end,fixes,assumes,shows,and,class,
    constrains , definition, where, apply, done,unfolding, primrec, fun, using, by, for, uses,
    schematic_lemma, concrete_definition, prepare_code_thms, export_code, datatype, type_synonym, typedef, value,
    proof, next, qed, show, have, hence, thus, interpretation, fix, context, sepref_definition,is
 } ,
  morekeywords=[2]{rec, return, bind, foreach, if, then, else, do, let, in, res, spec, fail, assert, while, case, of,
    check},
%  morekeywords=[3]{length,mod,insert},
%   morekeywords=[4]{simp,auto,intro,elim,rprems,refine_mono,refine_rcg},
  sensitive=True,
  morecomment=[s]{(\*}{\*)},
}

\lstset{
    language=isabelle,
    mathescape=true,
    escapeinside={--"}{"},
    basicstyle={\itshape},
    keywordstyle=\rm\bfseries,
    keywordstyle=[2]\rm\tt,
    keywordstyle=[3]\rm,
    keywordstyle=[4]\rm,
    showstringspaces=false,
    keepspaces=true,
    columns=[c]fullflexible}
\lstset{literate=
  {"}{}0
  {'}{{${}^\prime\!$}}1
  {\%}{{$\lambda$}}1
  {\\\%}{{$\lambda$}}1
  {\\\$}{{$\mathbin{\,\$\,}$}}1
  {->}{{$\rightarrow$}}1
  {<-}{{$\leftarrow$}}1
  {<.}{{$\langle$}}1
  {.>}{{$\rangle$}}1
  {<=}{{$\le$}}1
  {>=}{{$\ge$}}1
  {<->}{{$\leftrightarrow$}}1
  {-->}{{$\longrightarrow$}}2
  {<-->}{{$\longleftrightarrow$}}1
  {=>}{{$\Rightarrow$}}1
  {==}{{$\equiv$}}2
  {==>}{{$\implies$}}2
  {<=>}{{$\Leftrightarrow$}}1
  {~=}{{$\ne$}}1
  {|}{{$\mid$}}1
  {-`}{{$\rightharpoonup$}}1
  {|`}{{$\restriction$}}1
  {!!}{{$\bigwedge$}}1
  {(}{{$($}}1
  {)}{{$)$}}1
  {\{}{{$\{$}}1
  {\}}{{$\}$}}1
  {[}{{$[$}}1
  {]}{{$]$}}1
  {[|}{{$\llbracket$}}1
  {|]}{{$\rrbracket$}}1
  {\\<lbrakk>}{{$\lsem$}}1
  {\\<rbrakk>}{{$\rsem$}}1
  {|-}{{$\vdash$}}1
  {|=}{{$\models$}}1
  {|->}{{$\mapsto$}}1
  {|_|}{{$\bigsqcup$}}1
  {...}{{$\dots$}}1
  {\\x}{{$\times$}}1
  {_0}{{${}_0$}}1
  {_1}{{${}_1$}}1
  {_2}{{${}_2$}}1
  {_3}{{${}_3$}}1
  {_4}{{${}_4$}}1
  {_5}{{${}_5$}}1
  {_6}{{${}_6$}}1
  {_7}{{${}_7$}}1
  {_8}{{${}_8$}}1
  {_9}{{${}_9$}}1
  {_L}{{${}_L$}}1
  {\\_n}{{${}_n$}}1
  {\\_i}{{${}_i$}}1
  {\\_j}{{${}_j$}}1
  {\\_x}{{${}_x$}}1
  {\\_y}{{${}_y$}}1
  {\\impl}{{${}_\dagger$}}1
  {^*}{{$^*$}}1
  {^k}{{$^k$}}1
  {^d}{{$^d$}}1
  {\\<^sup>*}{{$^*$}}1
  {\\<^sub>*}{{$_*$}}1
  {\\<^sub>A}{{$_A$}}1
  {\\<^sub>r}{{$_r$}}1
  {\\<^sub>a}{{$_a$}}1
  {:_i}{{$:_i$}}1
  {\\<A>}{{$\mathcal{A}$}}1
  {\\<O>}{{\sf o}}1
  {\\<Phi>}{{$\Phi$}}1
  {\\<Psi>}{{$\Psi$}}1
  {\\<sigma>}{{$\sigma$}}1
  {\\<cdot>}{{$\cdot$}}1
  {\\<in>}{{$\in$}}1
  {\\<le>}{{$\le$}}1
  {\\<noteq>}{{$\ne$}}1
  {\\<lambda>}{{$\lambda$}}1
  {\\<longrightarrow>}{{$\longrightarrow$}}1
  {\\<longleftrightarrow>}{{$\longleftrightarrow$}}1
  {\\<Rightarrow>}{{$\Rightarrow$}}1
  {\\<Longrightarrow>}{{$\Longrightarrow$}}1
  {\\<rightarrow>}{{$\rightarrow$}}1
  {\\<leftarrow>}{{$\leftarrow$}}1
  {\\<mapsto>}{{$\mapsto$}}1
  {\\<equiv>}{{$\equiv$}}1
  {\\<and>}{{$\and$}}1
  {\\<or>}{{$\vee$}}1
  {\\<And>}{{$\bigwedge$}}1
  {\\<Up>}{{$\Uparrow$}}1
  {\\<Down>}{{$\Downarrow$}}1
  {\\<Union>}{{$\bigcup$}}1
  {\\<up>}{{$\uparrow$}}1
  {\\<down>}{{$\downarrow$}}1
  {\\<times>}{{$\times$}}1
  {\\<forall>}{{$\forall$}}1
  {\\<exists>}{{$\exists$}}1
  {\\<nexists>}{{$\nexists$}}1
  {\\<union>}{{$\cup$}}1
  {\\<inter>}{{$\cap$}}1
  {\\<subset>}{{$\subset$}}1
  {\\<subseteq>}{{$\subseteq$}}1
  {\\<supset>}{{$\supset$}}1
  {\\<supseteq>}{{$\supseteq$}}1
  {\\<alpha>}{{$\alpha$}}1
  {\\<beta>}{{$\beta$}}1
  {\\<gamma>}{{$\gamma$}}1
  {\\alpha}{{$\alpha$}}1
  {\\beta}{{$\beta$}}1
  {\\gamma}{{$\gamma$}}1
  {\\<Gamma>}{{$\Gamma$}}1
  {\\<langle>}{{$\langle$}}1
  {\\<rangle>}{{$\rangle$}}1
  {\\<not>}{{$\neg$}}1
  {\\<box>}{{$\oblong$}}1
  {\\<bot>}{{$\bot$}}1
  {\\<top>}{{$\top$}}1
  {\\<notin>}{{$\notin$}}1
  {\\<guillemotright>}{{$\gg$}}1
  {\\in}{$\in$}1
  {\\and}{$\wedge$}1
  {\\or}{$\vee$}1
  {\\Phi}{{$\Phi$}}1
  {\\Psi}{{$\Psi$}}1
  {\\le}{{$\le$}}1
  {\\Up}{{$\Uparrow$}}1
  {\\Down}{{$\Down$}}1
  {>>}{{$\gg$}}1
  {>>=}{{${\gg}{=}$}}1
  {<*lex*>}{{$\times_{\sf lex}$}}1
  {\\<open>}{{\rm\guilsinglleft}}1
  {\\<close>}{{\rm\guilsinglright}}1
}

% \newcommand{\is}{\lstinline[language=isabelle,basicstyle=\normalsize\ttfamily\slshape]}
\newcommand{\is}{\lstinline[language=isabelle]}
\newcommand{\q}[1]{\mbox{\guilsinglleft{#1}\hspace{-2.5pt}\guilsinglright}}
% \newcommand{\isai}[1]{\q{\lstinline[language=isabelle,basicstyle=\normalsize\ttfamily\slshape]{#1}}}
\cMakeRobust\q

\newcommand{\isai}{\lstinline[language=isabelle,basicstyle=\normalsize\ttfamily\slshape]}

\newcommand\CC{C\nolinebreak[4]\hspace{-.05em}\raisebox{.3ex}{\relsize{-1}{\textbf{++}}}}

\newcommand{\eqdef}{\mathrel{{=}_{def}}}
\newcommand{\iffdef}{\mathrel{{\mathord{\iff}\!\!}_{def}}}


\makeatletter
\newcommand*{\overlaynumber}{\number\beamer@slideinframe}
\makeatother

\AtBeginSection[] % Do nothing for \section*
{
  \begin{frame}<beamer>
    \frametitle{Outline}
    \tableofcontents[currentsection]
  \end{frame}
}


\title{The Isabelle Refinement Framework}

% \subtitle
% {Subtitle} % (optional)

\author[Peter Lammich]{Peter Lammich}
% - Use the \inst{?} command only if the authors have different
%   affiliation.

\institute[UoM] % (optional, but mostly needed)
{ University of Twente}
% - Use the \inst command only if there are several affiliations.
% - Keep it simple, no one is interested in your street address.

\date {May 2021}
% {2008-12-01}


% If you have a file called "university-logo-filename.xxx", where xxx
% is a graphic format that can be processed by latex or pdflatex,
% resp., then you can add a logo as follows:

% \pgfdeclareimage[height=0.5cm]{university-logo}{university-logo-filename}
% \logo{\pgfuseimage{university-logo}}


% Delete this, if you do not want the table of contents to pop up at
% the beginning of each subsection:


% If you wish to uncover everything in a step-wise fashion, uncomment
% the following command: 

%\beamerdefaultoverlayspecification{<+->}

%\mathchardef\-="2D
%\renewcommand\-{\text{-}}

\newcommand{\mc}{\color{blue}}
\newcommand{\term}[1]{{\mc#1}}

\let\olddisplaystyle\displaystyle
\newcommand{\mydisplaystyle}{\olddisplaystyle\mc}
\let\displaystyle\mydisplaystyle

\newcommand{\smc}{\everymath{\mc}}
\smc

\lstset{basicstyle=\color{blue}}

% \newcommand<>{\btikzset}[2]{\alt#3{\tikzset{#1}}{\tikzset{#2}}}

\tikzset{onslide/.code args={<#1>#2}{%
  \only<#1>{\pgfkeysalso{#2}} % \pgfkeysalso doesn't change the path
}}

\tikzset{uncover/.code args={<#1>#2}{%
  \uncover<#1>{\pgfkeysalso{#2}} % \pgfkeysalso doesn't change the path
}}


\tikzset{>=latex}


\lstset{autogobble}

\newcommand{\natN}{{\text{nat}_{\mathord{<}N}}}

\newcommand{\high}[1]{{\color{blue}#1}}

\begin{document}
% \input{macros}

\begin{frame}
  \titlepage
\end{frame}


\newcommand{\insertsectitle}{}

\setbeamertemplate{frametitle}{\vspace{.7em}\insertframetitle\hfill \small\raisebox{10pt}{\insertsectitle}}
# Introduction
  *<+-> Peter Lammich
    * new assistant professor in FMT group
      * previously in M\"unster, Munich, Virginia Tech, Manchester
    * research: software verification

  *<+-> if I'm not working: you'll probably find me rock-climbing
    *<+-> but I also enjoy hiking, biking (mtb, road, trek), racket sports (squash, badminton), ...


# The Sloth, HVS 5a, at the Roaches in Peak District
  \includegraphics[width=\textwidth]{sloth.jpg}

# Bull's Crack, HVS 5a, at Heptonstall
  \includegraphics[height=\textheight]{bullscrack.jpg}

# Sport Climbing (somewhere in the Peaks)
  \includegraphics[height=\textheight]{sportclimbing.jpg}

# Mountainbiking (at Lake Garda, after TransAlp)
  \includegraphics[height=\textheight]{mtb.jpg}

# Hiking in the Alps
  \includegraphics[height=\textheight]{hiking.jpg}


# ... and now to the serious part: Software Verification

  *<+-> Desirable properties of software
    *<+-> \onslide<.->{correct}\onslide<+->{ (formally verified)}
    *<+-> fast
    *<+-> manageable implementation\only<+->{ and proof} effort
  *<+-> Choose two!
  *<+-> This talk: towards faster verified algorithms at manageable effort

\def\algoAugment{
  \begin{algorithmic}
    \Procedure{augment}{$g,f,p$}
      \State{$c_p \gets \min\{ g_f(u,v) \mid (u,v)\in p \}$}
      \ForAll{$(u,v)\in p$}
        \If{$(u,v)\in g$}~$f(u,v)\gets f(u,v) + c_p$
        \Else~$f(v,u)\gets f(v,u) - c_p$
        \EndIf
      \EndFor
      \Return{$f$}
    \EndProcedure
  \end{algorithmic}
}

\def\algoEdka{
  \begin{algorithmic}
    \Procedure{Edmonds-Karp}{$g,s,t$}
      \State{$f\gets \lambda(u,v).~0$}
      \While{exists augmenting path in $g_f$}
        \State{$p \gets {}$ shortest augmenting path}
        \State{$f \gets $\Call{augment}{$g,f,p$}}
      \EndWhile
    \EndProcedure
  \end{algorithmic}
}

# Introduction
  *<+-> What does it need to formally verify an algorithm?
    *<+-> E.g. maxflow algorithms

  \uncover<+->{
  \begin{minipage}{.6\textwidth}
  \small\color{black}
  \algoAugment
  \algoEdka
  \end{minipage}%
  \begin{minipage}{.39\textwidth}\hspace*{-2em}
  \vspace*{-3em}
  \begin{tikzpicture}[node distance = 2cm, auto]
    \definecolor{capacity}{rgb}{0,0,0}
    \definecolor{flow}{rgb}{0,0,.7}

    \tikzstyle{flow}=[text=flow,swap]
    \tikzstyle{dflow}=[color=flow, draw opacity=.5]
    \tikzstyle{aug}=[color=red, draw opacity=.5, line width = 2pt]
    \tikzstyle{lab}=[text=capacity]

    \tikzstyle{nd}=[shape=circle,draw, minimum size = 7mm]

    \node[nd] (s) {s};
    \node[nd] (a) [above right of = s] {a};
    \node[nd] (b) [below right of = s] {b};
    \node[nd] (c) [right of = a] {c};
    \node[nd] (d) [right of = b] {d};
    \node[nd] (t) [below right of = c] {t};

    \draw[->] (s) to node[lab] {3} (a);
    \draw[->] (s) to node[lab] {5} (b);
    \draw[->] (a) to node[lab] {6} (c);
    \draw[->] (b) to node[lab] {3} (d);
    \draw[->] (c) to node[lab] {4} (t);
    \draw[->] (d) to node[lab] {4} (t);

    \draw[->] (a) to node[lab] {3} (d);

    \draw[dflow,line width = 3pt] (s) to node[flow] {3} (a);
    \draw[dflow,line width = 3pt] (a) to node[flow] {3} (d);
    \draw[dflow,line width = 3pt] (d) to node[flow] {3} (t);

    \draw[aug] (s) to (b);
    \draw[aug] (b) to (d);
    \draw[aug] (d) to (t);

  \end{tikzpicture}
  \end{minipage}

  \vfill

  $g$: flow network \hfill $s,t$: source, target \hfill $g_f$: residual network

  }


#[t] Correctness

  {\small
    \algoEdka
  }

  \vfill


  \only<2-4>{
    \begin{theorem}[Ford-Fulkerson]
      For a flow network $g$ and flow $f$, the following 3 statements are equivalent
      \begin{enumerate}
        \item $f$ is a maximum flow
        \item the residual network $g_f$ contains no augmenting path
        \item $|f|$ is the capacity of a (minimal) cut of $g$
      \end{enumerate}
    \end{theorem}
    \uncover<3-4>{
      \begin{proof}
        a few pages of definitions and textbook proof (e.g. Cormen).

        \uncover<4>{using basic concepts such as numbers, sets, and graphs.}
      \end{proof}
    }
  }

  \only<5-7>{
    \begin{theorem}
      Let $\delta_f$ be the length of a shortest $s,t$ - path in $g_f$.

      When augmenting with a shortest path,
      \begin{itemize}
        \item either $\delta_f$ decreases
        \item $\delta_f$ remains the same, and the number of edges in $g_f$ that lie on a shortest path decreases.
      \end{itemize}
    \end{theorem}
    \uncover<6-7>{
      \begin{proof}
        two more textbook pages.

        \uncover<7>{using lemmas about graphs and shortest paths.}
      \end{proof}
    }
  }

# Background Theory
  *<1-> E.g. graph theory
  *<2-> Typically requires powerful (interactive) prover
    * with good library support (to not re-invent too many wheels)

  *<3-> we use Isabelle
      *<4-> Isabelle/HOL: based on Higher-Order Logic
      *<5-> powerful automation (e.g. sledgehammer)
      *<6-> large collection of libraries
      *<7-> Archive of Formal Proofs
      *<8-> mature, production quality IDE, based on JEdit

  \vfill

  \hfill
  \uncover<4->{\includegraphics[width=.1\textwidth]{isabelle_hol.eps}}\hfill
  \uncover<5->{\includegraphics[width=.1\textwidth]{sledgehammer.png}}\hfill
  \uncover<6->{\includegraphics[width=.1\textwidth]{books.png}}\hfill
  \uncover<7->{\includegraphics[width=.1\textwidth]{isabelle_afp.eps}}\hfill
  \uncover<8->{\includegraphics[width=.1\textwidth]{JEdit_Logo.png}}\hfill{}


#! Implementation
  \begin{minipage}{.49\textwidth}
  \footnotesize
  \algoEdka
  \end{minipage}%
  \begin{minipage}{.49\textwidth}
  \tiny
  \begin{lstlisting}[language=C++]
    int edmonds_karp(int s, int t) {
        int flow = 0;
        vector<int> parent(n);
        int new_flow;

        while (new_flow = bfs(s, t, parent)) {
            flow += new_flow;
            int cur = t;
            while (cur != s) {
                int prev = parent[cur];
                capacity[prev][cur] -= new_flow;
                capacity[cur][prev] += new_flow;
                cur = prev;
            }
        }

        return flow;
    }
    \end{lstlisting}
  \end{minipage}



  \onslide<+->
  textbook proof typically covers abstract algorithm.

  \onslide<+->
  but this is quite far from implementation. Still missing:

  *<+-> optimizations: e.g., work on residual network instead of flow
  *<+-> algorithm to find shortest augmenting path (BFS)
  *<+-> efficient data structures: adjacency lists, weight matrix, FIFO-queue, ...
  *<+-> code extraction

#! Keeping it Manageable

  *<+-> A manageable proof needs modularization:
    *<+-> Prove separately, then assemble

  *<+-> Formal framework: Refinement
    *<+-> e.g.\ implement BFS, and prove it finds shortest paths
    *<+-> insert implementation into \textsc{EdmondsKarp}
  *<+-> Data refinement
    *<+-> BFS implementation uses adjacency lists. \textsc{EdmondsKarp} used abstract graphs.
    *<+-> refinement relations between
        * nodes and int64s (\high{node$_{64}$});
        * adjacency lists and graphs (\high{adjl});
        * arrays and paths (\high{array}).
      \onslide<+->
      \begin{lstlisting}
        (s$_\dagger$,s) \<in> node$_{64}$; (t$_\dagger$,t) \<in> node$_{64}$; (g$_\dagger$,g) \<in> adjl
        ==> (bfs s$_\dagger$ t$_\dagger$ g$_\dagger$, find_shortest s t g) \<in> array
      \end{lstlisting}

      \onslide<+->
      Shortcut notation: \is{(bfs,find_shortest) \<in> node$_{64}$->node$_{64}$->adjl->array}
  *<+-> Implementations used for different parts must fit together!

# Refinement Architecture (simplified)
  \begin{tikzpicture}[auto]

    \tikzstyle{eannot}=[font=\small, align=left]

    \node[visible on=<2->] (fsp) {shortest-path-spec};
    \node[visible on=<3->] (bfs1) [below = of fsp] {bfs-1};
    \node[visible on=<5->] (bfs) [below = of bfs1] {bfs};

%     \draw [->] (fsp) -- node [label=right:{''textbook'' proof}] {} -- (bfs1);

    \draw[visible on=<3->,->] (fsp) to node[eannot,visible on=<4->] (l1) {''textbook'' proof} (bfs1);
    \draw[visible on=<5->,->] (bfs1) to node[eannot,visible on=<6->] (l2) {graph $\to$ adj.-list\\queue $\to$ ring-buffer} (bfs);

    \node[visible on=<7->,fit=(fsp)(bfs1)(bfs)(l1)(l2), shape=rectangle, draw] (boxbfs) {};

    \node[visible on=<8->] (cmf) [right = 3.5cm of fsp] {maxflow-spec};
    \node[visible on=<9->] (edka1) [below = of cmf] {EdmondsKarp-1};
    \node[visible on=<11->] (edka2) [below = of edka1] {EdmondsKarp-2};
    \node[visible on=<13->] (edka) [below = 2cm of edka2] {EdmondsKarp};

    \draw[visible on=<9->, ->] (cmf) to node[eannot,visible on=<10->] (l3) {''textbook'' proof} (edka1);
    \draw[visible on=<11->, ->] (edka1) to node[eannot,visible on=<12->] (l4) {modify residual graph} (edka2);
    \draw[visible on=<13->, ->] (edka2) to node[eannot,visible on=<14->] (usebfs) {node $\to$ int\\graph $\to$ adj.-list\\capacity,flow $\to$ array\\shortest-path $\to$ bfs} (edka);

    \draw[visible on=<14->,->,dashed] (boxbfs) to[-|-] (usebfs.west);


    \only<15> {
    \overlaynodedrawfill[align=center]<
            {callout absolute pointer=(l1)},
            {callout absolute pointer=(l3)},
            {callout absolute pointer=(l4)}
        >
        [rectangle callout, below = of boxbfs]
        {substantial ideas \\ requires interactive proof}
    ;
    }

    \only<16> {
    \overlaynodedrawfill[align=center]<
            {callout absolute pointer=(l2)},
            {callout absolute pointer=(usebfs)}
        >
        [rectangle callout, below = of boxbfs]
        {straightforward \\ mainly automatic}
    ;
    }



  \end{tikzpicture}


# The Isabelle Refinement Framework
  *<+-> Formalization of Refinement in Isabelle/HOL
  *<+-> Tools + Automation
  *<+-> Libraries
  *<+-> Down to Ocaml/Haskell/Scala/SML and LLVM

#! IRF Core
  *<+-> Nondetermistic programs shallowly embedded in HOL
    * As monad
      \begin{lstlisting}
        \alpha M = FAIL | SPEC (\alpha => bool)
      \end{lstlisting}
      return, bind
    *<+-> + if-then-else, recursion (via flat ccpo)
    *<+-> + derived constructs (while, foreach, ...)
    *<+-> = usable programming language
  *<+-> Refinement Calculus for Program and Data Refinement
  *<+-> Automation: VCG, semi-automatic data refinement

# Imperative-HOL Backend
  * imperative + functional language
  * code generation to Ocaml/Haskell/Scala/SML
  * automatic refinement of functional to imperative DS
    * if used linearly

# Isabelle-LLVM Backend
  * only imperative + bounded integers
  * automatic placement of destructors
  * semi-automatic in-bound proofs (eg for int $\to$ int64)

  \includegraphics[height=.5\textheight]{isabelle-llvm.png}

# Refinement with Time
  * Prove correctness and complexity
  * \emph{Resource currencies} to structure complexity proofs along refinement
  * Down to Imperative-HOL / LLVM

  \includegraphics[height=.5\textheight]{isabelle-llvm-time.png}

# Libraries
  * Functional and Imperative data structures
    * readily usable for your developments

  * Functional:
    * hashtable, red-black-trees, tries, Finger-Trees, (Skew) binomial queues, ...

  * Imperative:
    * dynarray, heap, matrix, linked-list, hashtable, bit-vector, union-find, ROBDDs, B-Trees, ...

  \includegraphics[height=.2\textheight]{books.png}


# Highlight Verifications
  *<+-> CAVA model checker
    * fully fledged LTL model checker
    * developed independently by 3 groups
    * newer development: MUNTA for timed automata
  *<+-> Maxflow: Edmonds-Karp and Push-Relabel
    * textbook-level abstract correctness proof
    * efficient implementation
  *<+-> GRAT: SAT-Solver verification tool
    * faster than unverified state-of-the-art tool drat-trim
  *<+-> Introsort + Pdqsort
    * verified correctness and complexity
    * on par with C++ impls from GNU libstdc++ and Boost

# Future Work
  * Concurrency
  * Consolidate frameworks and tools
  * Interesting algorithms to verify

# Conclusions
  Isabelle Refinement Framework
    *{}: powerful interactive theorem prover
    *+: stepwise refinement
    *+: libraries for standard DS
    *+: lot's of automation
    *+: efficient backend (LLVM)
    *=: verified and efficient algorithms, at manageable effort


% # The Isabelle Refinement Framework
%   *<+-> Formalization of Refinement in Isabelle/HOL
%   *<+-> Batteries included
%     *<+-> Verification Condition Generator
%     *<+-> Collection Framework
%     *<+-> (Semi)automatic data refinement
%   *<+-> Some highlights
%     *<+-> GRAT UNSAT certification toolchain
%       * formally verified
%       * faster than (verified and unverified) competitors
%     *<+-> Introsort (on par with libstd++ std::sort)
%     *<+-> Timed Automata model checker
%     *<+-> CAVA LTL model checker
%     *<+-> Network flow (Push-Relabel and Edmonds Karp)
%
%
% # Formalizing Refinement
%   *<+-> Formal model for algorithms
%     * Require: nondeterminism, pointers/heap, (data) refinement
%     * VCG, also for refinements
%     * can get very complex!
%   *<+-> Current approach:
%     *# NRES: nondeterminism error monad with refinement ... but no heap
%       * simpler model, usable tools (e.g. VCG)
%     * HEAP: deterministic heap-error monad
%       * separation logic based VCG
%   *<+-> Automated transition from NRES to HEAP
%     * automatic data refinement (e.g. integer by int64)
%     * automatic placement on heap (e.g. list by array)
%     * some in-bound proof obligations left to user
%
%
%
% # Code Generation
%   Translate HEAP to compilable code
%   \onslide<+->
%
%   \begin{minipage}{.69\textwidth}
%   *#<+-> Imperative-HOL:
%     * based on Isabelle's code generator
%     * OCaml,SML,Haskell,Scala (using imp. features)
%     * results cannot compete with optimized C/C++
%   *#<+-> NEW!: Isabelle-LLVM
%     * shallow embedding of fragment of LLVM-IR
%     * pretty-print to actual LLVM IR text
%     * then use LLVM optimizer and compiler
%     * faster programs
%     * thinner (unverified) compilation layer
%   \end{minipage}%
%   \begin{minipage}{.3\textwidth}
%     \uncover<3->{
%     \vfill
%     \includegraphics[width=3cm]{isabelle-llvm.png}}
%   \end{minipage}
%
%
% # Knuth Morris Pratt
%   \begin{tikzpicture}
%     \begin{semilogyaxis}[
%       xlabel={Benchmark Set},
%       xlabel near ticks,
%       ylabel={Time/ms},
%       legend style = {
%         legend pos=outer north east,
%         cells={anchor=west}
%       },
%       ybar=0pt,
%       ymin=0, ymax=1e4,
%       bar width=4pt,
%       symbolic x coords={16-8,16-64,16-512,32-8,32-64,32-512,64-8,64-64,64-512},
%       xtick=data,
% %       nodes near coords,
% %       nodes near coords align={vertical},
%       x tick label style={rotate=45,anchor=east},
%     ]
%
%       \addplot[color=blue,fill=blue] coordinates {
%         (16-8,   4616)
%         (16-64,  4621)
%         (16-512, 4573)
%         (32-8,   4471)
%         (32-64,  4523)
%         (32-512, 4456)
%         (64-8,   4433)
%         (64-64,  4514)
%         (64-512, 4411)
%       };
%       \addlegendentry{Imperative HOL};
%
%       \addplot[color=green,fill=green] coordinates {
%         (16-8,   597)
%         (16-64,  598)
%         (16-512, 590)
%         (32-8,   551)
%         (32-64,  552)
%         (32-512, 544)
%         (64-8,   530)
%         (64-64,  531)
%         (64-512, 523)
%       };
%       \addlegendentry{LLVM};
%
%       \addplot[color=red,fill=red] coordinates {
%         (16-8,   499)
%         (16-64,  511)
%         (16-512, 513)
%         (32-8,   453)
%         (32-64,  465)
%         (32-512, 463)
%         (64-8,   418)
%         (64-64,  420)
%         (64-512, 416)
%       };
%       \addlegendentry{C++ (unverified)};
%
%
%     \end{semilogyaxis}
%   \end{tikzpicture}
%
%   Execute $a$-$l$ benchmark set from StringBench. Stop at first match.
%
%
% # Verified Sorting Algorithms: \only<1>{Introsort}\only<2>{Pdqsort}
%   \begin{tikzpicture}
%     \begin{axis}[
% %       xlabel={Benchmark Set},
%       xlabel near ticks,
%       ylabel={Time/ms},
%       legend style = {
%         legend pos=outer north east,
%         cells={anchor=west},
%         font=\scriptsize
%       },
%       ybar=0pt,
% %      ymin=0,
% %       ymax=1e4,
%       bar width=1.5pt,
%       symbolic x coords={rev-sorted-end-10,rev-sorted-end-1,sorted-end-.1,almost-sorted-50,random-boolean,organ-pipe,sorted-end-10,equal,rev-sorted-middle-.1,rev-sorted,sorted-middle-1,rev-sorted-middle-10,random,almost-sorted-.1,sorted,rev-sorted-middle-1,sorted-middle-.1,almost-sorted-10,almost-sorted-1,sorted-middle-10,rev-sorted-end-.1,sorted-end-1,random-dup-10,},      xtick=data,
% %       nodes near coords,
% %       nodes near coords align={vertical},
%       x tick label style={rotate=45,anchor=east,font=\tiny},
% %       ytick={1e3,1e4},
%       ymin=0,
%       ymax=1e4,
%       restrict y to domain=0:1e4,
% %       ytick distance = 10,
% %       ytickten = {1,2,3,4,5},
%     ]
%       \only<1>{
%       \addplot[color=red,fill=red] coordinates {
%       (rev-sorted-end-10,3827)
%       (rev-sorted-end-1,4186)
%       (sorted-end-.1,4781)
%       (almost-sorted-50,6477)
%       (random-boolean,1667)
%       (organ-pipe,8457)
%       (sorted-end-10,6043)
%       (equal,1142)
%       (rev-sorted-middle-.1,3342)
%       (rev-sorted,981)
%       (sorted-middle-1,3799)
%       (rev-sorted-middle-10,5527)
%       (random,7368)
%       (almost-sorted-.1,1419)
%       (sorted,1438)
%       (rev-sorted-middle-1,3302)
%       (sorted-middle-.1,3448)
%       (almost-sorted-10,3107)
%       (almost-sorted-1,1590)
%       (sorted-middle-10,5514)
%       (rev-sorted-end-.1,6211)
%       (sorted-end-1,9103)
%       (random-dup-10,8175)
%       };
%       \addlegendentry{Isabelle-LLVM};
%       \addplot[color=blue,fill=blue] coordinates {
%       (rev-sorted-end-10,3787)
%       (rev-sorted-end-1,3815)
%       (sorted-end-.1,4982)
%       (almost-sorted-50,6480)
%       (random-boolean,1669)
%       (organ-pipe,9203)
%       (sorted-end-10,6312)
%       (equal,1318)
%       (rev-sorted-middle-.1,3544)
%       (rev-sorted,963)
%       (sorted-middle-1,3376)
%       (rev-sorted-middle-10,5460)
%       (random,7405)
%       (almost-sorted-.1,1376)
%       (sorted,1349)
%       (rev-sorted-middle-1,3380)
%       (sorted-middle-.1,3548)
%       (almost-sorted-10,2994)
%       (almost-sorted-1,1740)
%       (sorted-middle-10,5574)
%       (rev-sorted-end-.1,5880)
%       (sorted-end-1,9601)
%       (random-dup-10,8190)
%       };
%       \addlegendentry{libstd++ (unverified)};
%       }
%       \only<2>{
%         \addplot[color=red,fill=red] coordinates {
%         (rev-sorted-end-10,4278)
%         (rev-sorted-end-1,2245)
%         (sorted-end-.1,1817)
%         (almost-sorted-50,6985)
%         (random-boolean,474)
%         (organ-pipe,3278)
%         (sorted-end-10,4008)
%         (equal,135)
%         (rev-sorted-middle-.1,2012)
%         (rev-sorted,268)
%         (sorted-middle-1,2603)
%         (rev-sorted-middle-10,6006)
%         (random,8066)
%         (almost-sorted-.1,891)
%         (sorted,200)
%         (rev-sorted-middle-1,2613)
%         (sorted-middle-.1,2015)
%         (almost-sorted-10,2982)
%         (almost-sorted-1,1204)
%         (sorted-middle-10,6005)
%         (rev-sorted-end-.1,1795)
%         (sorted-end-1,2278)
%         (random-dup-10,7767)
%         };
%         \addlegendentry{Isabelle-LLVM};
%
%         \addplot[color=blue,fill=blue] coordinates {
%         (rev-sorted-end-10,4072)
%         (rev-sorted-end-1,2215)
%         (sorted-end-.1,1816)
%         (almost-sorted-50,6606)
%         (random-boolean,458)
%         (organ-pipe,3173)
%         (sorted-end-10,3840)
%         (equal,148)
%         (rev-sorted-middle-.1,1992)
%         (rev-sorted,230)
%         (sorted-middle-1,2530)
%         (rev-sorted-middle-10,5673)
%         (random,7505)
%         (almost-sorted-.1,849)
%         (sorted,154)
%         (rev-sorted-middle-1,2529)
%         (sorted-middle-.1,1992)
%         (almost-sorted-10,2844)
%         (almost-sorted-1,1123)
%         (sorted-middle-10,5673)
%         (rev-sorted-end-.1,1761)
%         (sorted-end-1,2260)
%         (random-dup-10,7220)
%         };
%         \addlegendentry{Boost (unverified)};
%       }
%     \end{axis}
%   \end{tikzpicture}
%
%   Sorting $100\cdot10^6$ \texttt{uint64}s on
%   Intel Core i7-8665U CPU, 32GiB RAM
%
% # Current (near Future) Projects
%   *<+-> Framework
%     *<+-> Scalable Sepref Tool
%     *<+-> Nested Containers
%     *<+-> Nice input language
%     *<+-> Support for Nres+Time
%   *<+-> Applications
%     *<+-> SAT
%       * Verified SAT Solver
%       * Verified drat-trim
%       * QBF certificate checking
%     *<+-> Graphs: Efficient Blossom Algorithm Implementation
%     *<+-> Sorting:
%       * Branch-aware partitioning
%       * Stable sorts
%   *<+-> ...
%
% #! Sepref Tool
%   *<+-> Synthesize imperative program from functional
%     * sep-logic assertion relating concrete with abstract variables
%
%     \begin{minipage}{.48\textwidth}
%     \onslide<+->
%     \begin{lstlisting}
%       f (l :: int list) {
%         (int)set S = {}
%         int c=0
%         for (int i=0; i<|l|; ++i) {
%           t_1 = l[i]
%           if (t_1 \<notin> S) {
%             ${}$*: assert (c<|l|)
%             ++c
%             S={t_1} \<union> S
%           } } }
%     \end{lstlisting}
%     \end{minipage}%
%     \begin{minipage}{.48\textwidth}
%     \onslide<+->
%     \begin{lstlisting}
%       f (l' :: int_6_4 array) {
%         hashmap S' = hm_empty()
%         int_6_4 c'=0
%         for (int_6_4 i'=0; i'<|l'|; ++i') {
%           t_1' = l[i']
%           if (\<not>hm_member t_1' S') {
%             ${}$*:
%             ++c'
%             S'=hm_insert t_1' S'
%           } } free S' }
%     \end{lstlisting}
%     \end{minipage}
%   \onslide<+->At *: \isai{ array i64 l l' * hm i64 S S' * ... }
%
% #! Nested Containers
%   \onslide<+->
%   Hoare-Rule for array-index:
%   \begin{lstlisting}
%   {array A l l' * i64 i i' * i<|l|} r'=l'[i'] { array A l l' * i64 i i' * A (l[i]) r' }
%   where
%   array A l p = \<exists> l'. p+0 |-> l'[0] * ... * p+n |-> l'[n]
%                      ${}$* A l[0] l'[0] * ... * A l[n] l'[n]
%   \end{lstlisting}
%
%   \onslide<+->
%   Problem: Does not work for \isai{array (array i64)}! (result is shared)
%   *<+-> current approach: abstract data type: \isai$\<alpha> option list$
%     * None: element not in array
%     * Manual ownership management
%   *<+-> future:
%     * read-only sharing (fractional sep-logic?)
%     * automation (as far as possible)
%     * maybe inspiration from Rust.
%
%
% %
% %   * Problem: implement list of lists of ints by array
% %     * \isai{ array_assn (array_assn int_assn) } where
% %       \begin{lstlisting}
% %         array_assn A l p = \<exists> l'. p+0 |-> l'_0 * ... * p+n |-> l'_n
% %                             * A l'_0 l_0 * ... * A l'_n l_n
% %           = A l_0 (p+0) * ... * A l_n (p+n)
% %       \end{lstlisting}
%
%
% # Conclusions
%   Isabelle Refinement Framework
%     *{}: powerful interactive theorem prover
%     *+: stepwise refinement
%     *+: libraries for standard DS and algorithms
%     *+: lot's of automation
%     *+: efficient backend (LLVM)
%     *=: verified and efficient algorithms, at manageable effort
%
%   \vfill
%
%   \url{https://github.com/lammich/isabelle_llvm}
%
%
%





%
% xxx, ctd here: Formal model for algorithms required!
%   that supports refinement, nondeterminism, imperative data structures (arrays)
%
%   ... not yet (completely) there
%   instead, using two-step process:
%     1. purely functional representation that supports nondeterminism and refinement
%     2. deterministic representation with imperative features (heap, pointers)
%
%   highly automated transition from 1 to 2:
%     automatically implement abstract concepts (e.g. matrix) by concrete ones (e.g. 2dim array)
%       or: BigInt by int64
%     automatically place linearly used data on heap, destructive update
%     sometimes, needs hints from user to discharge additional proof obligations (e.g. indexes in bounds, no overflows)
%
%
%   Finally: translate 2 to actual runnable program
%     * Isabelle/HOL code generator -> OCaml, SML, Haskell, Scala
%       * using imperative features of these languages
%       * -> fast, but cannot compete with optimized C/C++ code!
%     * NEW! Isabelle-LLVM -> LLVM IR
%       * -> we're getting there!
%       * and thinner (unverified) compiler layer
%
%   ** show sorting benchmarks **
%
%   ** then continue with Isabelle-LLVM talk




%
% # Overview
%
%   \center
%   \begin{tikzpicture}[
%       BX/.style={rectangle,draw,align=center,outer sep=0,inner sep=0,minimum height=1cm,node distance=1cm, anchor=south west,transform shape,text=lightgray},
%       TCB/.style={onslide=<4->{fill=green!30,text=black}},
%       basic layer/.style={onslide=<3->{fill=blue!30,text=black}},
%       FE/.style={onslide=<2->{fill=red!30,text=black}},
% %       APP/.style={onslide=<2->{text=black}},
%       scale=.8
%     ]
%
%     \node [BX,text width=10cm,text=black] at (0,0) {LLVM Code};
%     \node [BX,text width=10cm, TCB] at (0,1) {Code Generator};
%     \node [BX,text width=10cm, TCB] at (0,2) {Semantics};
%     \node [BX,text width=5cm, minimum height=3cm, basic layer] at (0,3) {Preprocessor};
%     \node [BX,text width=5cm, basic layer] at (5,3) {Separation Logic};
%     \node [BX,text width=5cm, minimum height=2cm, basic layer] at (5,4) {\\[1cm]VCG};
%     \node [BX,text width=5cm,fill=white, basic layer] at (2.5,5) {Basic Data Structures};
%     \node [BX,text width=10cm, FE] at (0,6) {Sepref Tool};
%     \node [BX,text width=5cm,FE] at (0,7) {Refinement Framework};
%     \node [BX,text width=5cm,FE] at (5,7) {Collection Framework};
%     \node [BX,text width=10cm,,text=black] at (0,8) {Verified Algorithms};
%
%     \node [label=left:{\color{white}\color<4->{green}Kernel (TCB)}] at (0,2) {};
%     \node [label=left:{\color{white}\color<3->{blue}Basic Layer}] at (0,4.5) {};
%     \node [label=left:{\color{white}\color<2->{red}Frontend}] at (0,7) {};
% %     \node [label=left:{\color{white}\color<2->{black}Applications}] at (0,8.5) {};
%
%
% %     \node [text width=5cm,thispaper] at (0,2) {Separation Logic};
% %     \node [text width=5cm, minimum height=2cm] at (5,1) {Refinement Framework};
% %     \node [text width=2.5cm,thispaper] at (0,3) {Imperative Collections};
% %     \node [text width=3.5cm,thispaper] at (2.5,3) {Refinement to Imp/HOL};
% %     \node [text width=2.5cm] at (6,3) {Functional Collections};
% %     \node [text width=1.5cm, minimum height=2cm] at (8.5,3) {Abstract Algorithms};
% %     \node [text width=8.5cm,thispaper] at (0,4) {Automatic Refinement};
% %     \node [text width=10cm,thispaper] at (0,5) {Efficient Imperative Algorithms};
%   \end{tikzpicture}
%
% %   \uncover<2->{\color{green}TCB}~~\uncover<3->{\color{blue}Basic Layer}~~\uncover<4->{\color{red}Frontend}
%
%
% %
% %   Diagram!
% %   * Trusted Code Base
% %     * Code generator (Pretty-printing + ctrl flow compiler)
% %     * LLVM semantics (based on state/error monad, memory model, shallow embedding)
% %
% %   * Tools
% %     * Preprocessor
% %     * VCG
% %     * Sepref (Link to Refinement Framework)
%
%
% \renewcommand{\insertsectitle}{\color{green}{Kernel}}
%
%
% # LLVM Semantics
%   *<+-> We don't need to formalize all of LLVM!
%     * just enough to express meaningful programs
%     * abstract away certain details (e.g. in memory model)
%   *<+-> Trade-off
%     * complexity of semantics vs. trusted steps in code generator
%   *<+-> Our choice:
%     * rather simple semantics
%     * code generator does some translations
%
% #! Basics
%   * LLVM operations described in state/error monad
%     \begin{lstlisting}
%     \<alpha> llM = llM (run: "memory => \<alpha> mres")
%     \<alpha> mres = NTERM | FAIL | SUCC \<alpha> memory
%     \end{lstlisting}
%
%     \pause
%
%       \begin{lstlisting}
%         ll_udiv :: "n word => n word => n word llM"
%         ll_udiv a b = do { assert (b ~= 0); return (a div b) }
%       \end{lstlisting}
%
%     \pause
%
%       \begin{lstlisting}
%         llc_if b t e = if b~=0 then t else e
%       \end{lstlisting}
%
%     \pause
%
%   * Recursion via fixed-point
%     \begin{lstlisting}
%       llc_while b f s_0 = fixp (%W s.
%           do {
%             ctd <- b s;
%             if ctd~=0 then do {s <- f s; W s} else return s
%           }
%         ) s_0
%     \end{lstlisting}
% %     \pause
% %     * partial-function: convenient definition via recursive equations
%
%
%
% %
% %   * Standard operations: \is{return, bind, fail, get, set, rec}
% %     * from HOL: \is{if, let}
% %     * derived: \is{assert, while}, ...
%
%
%
% % # Shallow Embedding
% %   *<+-> LLVM-program: set of function definition theorems
% %     * arbitrary recursive functions definable (fixed-points in monad)
% %   *<+-> Argument and return types
% %     * pointers, pairs, \is{n}-bit words
% %     * monomorphic
% %   *<+-> Only contains instructions and other defined functions
% %     * only variable and constant parameters (flattened expressions)
% %   *<+-> Code generator
% %     * Pretty-print and compile control flow
% %
%
%
% %     \begin{lstlisting}
% %     program = (thm: proc_name args = block)^*
% %     block = do { var <- cmd; block } | return var
% %     cmd = ll_<opcode> arg^* | proc_name arg^* | llc_if arg block block
% %     \end{lstlisting}
% %
% %     * arguments must be variables or constants
% %     * definitions must be monomorphic
% %     * variable and return types of shape
% %       * \is{V = V ptr | V\<times>V | n word}
% %       * pointers, pairs, integers of width \is{n}
% %
% %   \pause
% %
% %
% %   * Semantics provides definitions for \is{ll_<opcode>} constants
% %     \begin{lstlisting}
% %       ll_udiv :: "n word => n word => n word llM"
% %       ll_udiv a b = do { assert (b ~= 0); return (a div b) }
% %     \end{lstlisting}
%
% #! Shallow Embedding
%   \newcommand{\mknd}[1]{\makebox[0pt]{\tikz[remember picture]{\node (#1) {};}}}
%   {\small\centering
%   \begin{minipage}[t]{.35\textwidth}
%     \begin{lstlisting}[escapechar=!]
%       fib:: 64 word !\mknd{ndparams}!=> 64 word ll!\mknd{ndrettype}!M
%       fib n = do {
%         t <- ll_icmp_ule n 1;
%         llc_if!\mknd{stmtif}! t
%           (return n)
%           (do {
%             n_1 <- ll_sub!\mknd{stmtsub}! n 1;
%             a   <- fib!\mknd{stmtrec}! n_1;
%             n_2 <- ll_sub n!\mknd{ndargs}! 2;
%             b   <-!\mknd{ndbind}! fib n_2;
%             c   <- ll_add a b;
%             return!\mknd{ndreturn}! c
%           }) }
%     \end{lstlisting}
% %     \begin{tikzpicture}[remember picture,overlay,ND/.style={rectangle callout, anchor=west,draw , fill=red!50, callout absolute pointer={#1}}]
% %       \node[ND=(ndrettype.east),anchor=south] at ($(ndrettype.north) + (0,.5cm)$) {\footnotesize state/error monad};
% %       \pause
% %       \node[ND=(ndparams.east)] at ($(ndparams -| 6,0) + (0,.5cm)$) {\footnotesize parameter types: words, pointers, pairs};
% %       \pause
% %       \node[ND=(stmtif.east)] at ($(stmtif -| 6,0) + (0,.5cm)$) {\footnotesize control flow (if, [optional: while])};
% %       \pause
% %       \node[ND=(stmtsub.east)] at ($(stmtsub -| 6,0)+(0,.5cm)$) {\footnotesize standard instructions};
% %       \pause
% %       \node[ND=(stmtrec.east)] at ($(stmtrec -| 6,0)+(0,.5cm)$) {\footnotesize function calls};
% %       \pause
% %       \node[ND=(ndargs.east)] at ($(ndargs -| 6,0)+(0,.5cm)$) {\footnotesize arguments: variables and constants};
% %       \pause
% %       \node[ND=(ndbind.east)] at ($(ndbind -| 6,0)+(0,.5cm)$) {\footnotesize monad: bind, return};
% %     \end{tikzpicture}
%     \begin{tikzpicture}[remember picture,overlay,ND/.style={anchor=west}]
%       \pause
%       \node[ND,anchor=south] at ($(ndrettype.north) + (0,.5cm)$) (X) {\footnotesize state/error monad};
%       \draw (X) -- (ndrettype);
%
%       \pause
%       \node[ND] at ($(ndparams -| 6,0) + (0,.5cm)$) (X) {\footnotesize types: words, pointers, pairs};
%       \draw (X.west) -- (ndparams);
%
%       \pause
%       \node[ND=(ndbind.east)] at ($(ndbind -| 6,0)+(0,.5cm)$) (X) {\footnotesize monad: bind, return};
%       \draw (X.west) -- (ndbind);
%       \draw (X.west) -- (ndreturn);
%
%       \pause
%       \node[ND=(stmtsub.east)] at ($(stmtsub -| 6,0)+(0,.5cm)$) (X) {\footnotesize standard instructions (\is{ll_<opcode>})};
%       \draw (X.west) -- (stmtsub);
%
%       \pause
%       \node[ND=(ndargs.east)] at ($(ndargs -| 6,0)+(0,.5cm)$) (X) {\footnotesize arguments: variables and constants};
%       \draw (X.west) -- (ndargs);
%
%       \pause
%       \node[ND=(stmtif.east)] at ($(stmtif -| 6,0) + (0,.5cm)$) (X) {\footnotesize control flow (if, [optional: while])};
%       \draw (X.west) -- (stmtif);
%
%       \pause
%       \node[ND=(stmtrec.east)] at ($(stmtrec -| 6,0)+(0,.5cm)$) (X) {\footnotesize function calls};
%       \draw (X.west) -- (stmtrec);
%
%     \end{tikzpicture}
%   \end{minipage}
%   }
%
%
%
%
%
% #! Code Generation
%   \uncover<2->{compiling control flow + pretty printing}
%
%   {\small\centering
%   \begin{minipage}[t]{.35\textwidth}
%     \begin{lstlisting}
%       fib:: 64 word => 64 word llM
%       fib n = do {
%         t <- ll_icmp_ule n 1;
%         llc_if t
%
%           (return n)
%           (do {
%             n_1 <- ll_sub n 1;
%             a   <- fib n_1;
%             n_2 <- ll_sub n 2;
%             b   <- fib n_2;
%             c   <- ll_add a b;
%             return c
%           }) }
%     \end{lstlisting}
%   \end{minipage}
%   % \hspace*{.04\textwidth}
%   \hfill\pause
%   \begin{minipage}[t]{.57\textwidth}
%     \begin{lstlisting}[language=LLVM, literate={}]
%     define i64 @fib(i64 %x) {
%       start:
%         %t = icmp ule i64 %x, 1
%         br i1 %t, label %then, label %else
%       then:
%         br label %ctd_if
%       else:
%         %n_1 = sub i64 %x, 1
%         %a    = call i64 @fib (i64 %n_1)
%         %n_2 = sub i64 %x, 2
%         %b    = call i64 @fib (i64 %n_2)
%         %c    = add i64 %a, %b
%         br label %ctd_if
%       ctd_if:
%         %x1a = phi i64 [%x,%then], [%c,%else]
%         ret i64 %x1a }
%     \end{lstlisting}
%   \end{minipage}
%   }
%
% #! Memory Model
%   * Inspired by CompCert v1. But with structured values.
%
%     \begin{lstlisting}
%       memory = block list      block = val list option
%       val = n word | ptr | val\<times>val
%       rptr = NULL | ADDR nat nat (dir list)      dir = FST | SND
%     \end{lstlisting}
%
%     * \is{ADDR i j p} block index, value index, path to value
%
%   \pause
%
%   * Typeclass \is{llvm_rep}: shallow to deep embedding
%     \begin{lstlisting}
%     to_val :: 'a => val
%     from_val :: val => 'a
%     init :: 'a  $\text{\color{darkgray} -- Zero initializer}$
%     \end{lstlisting}
%
%   \pause
%
%   * Shallow pointers carry phantom type
%     \begin{lstlisting}
%       'a ptr = PTR rptr
%     \end{lstlisting}
%
%
% #! Example: malloc
%   \begin{lstlisting}
%     allocn (v::val) (s::nat) = do {
%       bs <- get;
%       set (bs@[Some (replicate s v)]);
%       return (ADDR |bs| 0 []) }
%   \end{lstlisting}
%   \pause
%
%   \begin{lstlisting}
%     ll_malloc (s::n word) :: 'a ptr = do {
%       assert (unat n > 0); $\text{\color{darkgray}-- Disallow empty malloc}$
%       r <- allocn (to_val (init::'a)) (unat n);
%       return (PTR r) }
%   \end{lstlisting}
%   \pause
%
%   * Code generator maps \is{ll_malloc} to libc's \is{calloc}.
%     * out-of-memory: terminate in defined way \is{exit(1)}
%
% \renewcommand{\insertsectitle}{\color{blue}{Basic Layer}}
%
% #! Preprocessor
%   * Restricted terms accepted by code generator
%     * good to keep code generation simple
%     * tedious to write manually
%   \pause
%   * Preprocessor transforms terms into restricted format
%       \pause
%     * proves equality (via Isabelle kernel)
%       \pause
%     * monomorphization (instantiate polymorphic definitions)
%       \pause
%     * flattening of expressions
%       \begin{lstlisting}
%         return ((a+b)+c) |-> do {t<-ll_add a b; ll_add t c}
%       \end{lstlisting}
%       \pause
%     * tuples
%       \begin{lstlisting}
%         return (a,b) |-> do { t<-ll_insert_1 init a; ll_insert_2 t b }
%       \end{lstlisting}
%       \pause
%     * Define recursive functions for fixed points
% %     * fixed-points: For \is"f" with \is"f = F f", \is"mono F"
% %       * Definition containing \is"f": define new recursive function
%
% #! Example: Preprocessing Euclid's Algorithm
%   \small
%   \begin{lstlisting}
%   euclid :: 64 word => 64 word => 64 word
%   euclid a b = do {
%     (a,b) <- llc_while
%       (%(a,b) => ll_cmp (a ~= b))
%       (%(a,b) => if (a<=b) then return (a,b-a) else return (a-b,b))
%       (a,b);
%     return a }
%   \end{lstlisting}
%   \pause
%   preprocessor defines function \is{euclid_0} and proves
%   \begin{lstlisting}
%   euclid a b = do {
%       ab <- ll_insert_1 init a; ab <- ll_insert_2 ab b;
%       ab <- euclid_0 ab;
%       ll_extract_1 ab  }
%   euclid_0 s = do {
%     a <- ll_extract_1 s;
%     b <- ll_extract_2 s;
%     ctd <- ll_icmp_ne a b;
%     llc_if ctd do {...; euclid_0 ...} }
%   \end{lstlisting}
%
% #! Reasoning about LLVM Programs
%   * Separation Logic
%     * Hoare-triples
%       \begin{lstlisting}
%         \<alpha> :: memory -> amemory :: sep_algebra
%         wp c Q s = \<exists>r s'. run c s = SUCC r s' \and Q r (\<alpha> s')
%         |= {P} c {Q} = \<forall>F s. (P*F) (\<alpha> s) --> wp c (\<lambda>r s'. (Q r * F) s') s
%       \end{lstlisting}
%
%     \pause
%     * memory primitives\\
%       \is"p|->x" -- \is"p" points to value \is"x"\\
%       \is"m_tag n p" -- ownership of block (not its contents)\\[1em]
%       \is"range {i_1,...,i_n} f p = (p+i_1)|->(f i_1) * ... * (p+i_n)|->(f i_n)"
%
%     \pause
%     * rules for commands
%       \begin{lstlisting}
%         b ~= 0 ==> |= {\<box>} ll_udiv a b {%r. r = a div b}
%         |= {n~=0} ll_malloc n {%p. range {0..<n} (%_. init) p * m_tag n p}
%         |= {p|->x} ll_load p {%r. r=x * p|->x}
%       \end{lstlisting}
%
%   \pause
%   * Automation: VCG, frame inference, heuristics to discharge VCs
%
%   \pause
%   * Basic Data Structures: signed/unsigned integers, Booleans, arrays
%
% #! Example: Proving Euclid's Algorithm
%   \small
%   \begin{lstlisting}[escapechar=!]
%     lemma
%     |= {uint$_{64}$ a a\impl * uint$_{64}$ b b\impl * 0<a * 0<b} euclid a\impl b\impl {%r\impl. uint$_{64}$ (gcd a b) r\impl}
%     !\pause!
%     unfolding euclid_def
%     apply (rewrite annotate_llc_while[where I = ... and R = measure nat])
%     !\pause!
%     apply (vcg; clarsimp?)
%   \end{lstlisting}
%   \pause
%
%   {\everymath{\color{darkgray}}
%   \begin{lstlisting}[basicstyle=\color{darkgray}]
%     Subgoals:
%       1. !!x y. [| gcd x y = gcd a b; x ~= y; x <= y; ... |] ==> gcd x (y - x) = gcd a b
%       2. !!x y. [| gcd x y = gcd a b; \<not> x <= y; ... |] ==> gcd (x - y) y = gcd a b
%   \end{lstlisting}
%   }
%   \pause
%
%   \begin{lstlisting}
%     by ( simp_all add: gcd_diff1 gcd_diff1' )
%   \end{lstlisting}
%
% % # Basic Data Structure Library
% %   *<+-> Use VCG to provide basic data structures
% %     * signed/unsigned integers
% %     * Booleans
% %     * arrays (fixed-size, dynamic)
% %
% %   *<+-> Used as building-blocks for advanced DS
% %     * formalized using Sepref
%
% \renewcommand{\insertsectitle}{\color{red}{Frontend}}
%
% # Automatic Refinement
%   *<+-> Isabelle Refinement Framework
%     * supports verification by stepwise refinement
%     * many verified algorithms already exists
%
%   *<+-> Sepref tool
%     * refinement from Refinement Framework to imperative program
%       * already existed for Imperative/HOL
%       * we adapted it for LLVM
%     * existing proofs can be re-used
%       * need to be amended if they use arbitrary-precision integers
%
%   *<+-> Collections Framework
%     * provides data structures
%     * we ported some to LLVM (work in progress)
%       * dense sets/maps of integers (by array)
%       * heaps, indexed heaps
%       * two-watched-literals for BCP
%       * graphs (by adjacency lists)
%       * ...
%
% %   *<+-> Sepref tool
% %     * automatic translation from Refinement Framework to LLVM
% %     * tools for data structure development
% %     * existing proofs can be re-used
% %       * need to be amended if they use arbitrary-precision integers
%
% %   *<+-> Advanced Data Structures
% %     * dense sets/maps of integers (by array)
% %     * heaps, indexed heaps
% %     * two-watched-literals for BCP
% %     * graphs (by adjacency lists)
% %     * ...
% %
% %   * TODO: Example: Synthesis process?
% %
% %   * TODO: Example: binsearch
%
% #! Example: Binary Search
%   \begin{lstlisting}
%     definition "bin_search xs x = do {
%       (l,h) <- WHILEIT (bin_search_invar xs x)
%         (%(l,h). l<h)
%         (%(l,h). do {
%           ASSERT (l<length xs \<and> h<=length xs \<and> l<=h);
%           let m = l + (h-l) div 2;
%           if xs!m < x then RETURN (m+1,h) else RETURN (l,m)
%         })
%         (0,length xs);
%       RETURN l
%     }"
%   \end{lstlisting}
%   \pause
%
%   \begin{lstlisting}
%   lemma bin_search_correct:
%     sorted xs ==> bin_search xs x <= SPEC (%i. i=find_index (%y. x<=y) xs)"
%   \end{lstlisting}
%
% #! Example: Binary Search --- Refinement
%
%   \begin{lstlisting}
%   sepref_def bin_search_impl is "uncurry bin_search"
%     :: "(larray_assn' TYPE(size_t) (sint_assn' TYPE(elem_t)))$^k$
%         $$* (sint_assn' TYPE(elem_t))$^k$
%        -> snat_assn' TYPE(size_t)"
%     unfolding bin_search_def
%     apply (rule hfref_with_rdomI, annot_snat_const "TYPE(size_t)")
%     by sepref
%   \end{lstlisting}
%   \pause
%
%   \begin{lstlisting}
%   export_llvm bin_search_impl is int64_t bin_search(larray_t, elem_t)
%   defines
%     typedef uint64_t elem_t;
%     typedef struct { int64_t len; elem_t *data; } larray_t;
%   file "code/bin_search.ll"
%   \end{lstlisting}
%
% #! Example: Binary Search --- Generated Code
%   Produces LLVM code and header file:
%   \begin{lstlisting}[language=C]
%     typedef uint64_t elem_t;
%     typedef struct {
%       int64_t len;
%       elem_t*data;
%     } larray_t;
%
%     int64_t bin_search(larray_t,elem_t);
%   \end{lstlisting}
%
%
% \renewcommand{\insertsectitle}{{Benchmarks}}
%
%
% # Case Studies
%   *<+-> Binary Search and Knuth-Morris-Pratt
%     * manageable amount of changes to original formalization
%
%   *<+-> Efficiency
%     * on par with unverified C/C++
%     * one order of magnitude faster than original
% %     * still 1.5 times faster than SML with fixed-size integers
%
% % \renewcommand{\insertsectitle}{}
%
% # Binary Search
%
%   \begin{tikzpicture}
%     \begin{loglogaxis}[
%       xlabel={Array Size ($n$)},
%       ylabel={Time/ms},
%       legend style = {
%         legend pos=outer north east,
%         cells={anchor=west}
%       }
%     ]
%
%
%       \addplot[color=blue,mark=*] coordinates {
%         (1e6, 1999 )
%         (2e6, 4209 )
%         (3e6, 6516 )
%         (4e6, 8843 )
%         (5e6, 11494)
%         (6e6, 13646)
%         (7e6, 16032)
%         (8e6, 18421)
%         (9e6, 20957)
%         (10e6,23409)
%       };
%       \addlegendentry{Imperative HOL};
%
%
%       \addplot[color=red,mark=square*] coordinates {
%         (1e6,  121 )
%         (2e6,  251 )
%         (3e6,  379 )
%         (4e6,  513 )
%         (5e6,  635 )
%         (6e6,  767 )
%         (7e6,  908 )
%         (8e6,  1038)
%         (9e6,  1162)
%         (10e6, 1293)
%       };
%       \addlegendentry{C (unverified)};
%
%       \addplot[color=green,mark=diamond*] coordinates {
%         (1e6, 100 )
%         (2e6, 204 )
%         (3e6, 304 )
%         (4e6, 412 )
%         (5e6, 514 )
%         (6e6, 617 )
%         (7e6, 726 )
%         (8e6, 854 )
%         (9e6, 945 )
%         (10e6,1045)
%       };
%       \addlegendentry{LLVM};
%
%
%     \end{loglogaxis}
%   \end{tikzpicture}
%
%   Search for the values $0,2,\ldots<5n$ in an array $[0,5,\ldots<5n]$
%
% %     $1$      & 121  & 100  & 1999   & 139      \\
% %     $2$      & 251  & 204  & 4209   & 289      \\
% %     $3$      & 379  & 304  & 6516   & 440      \\
% %     $4$      & 513  & 412  & 8843   & 600      \\
% %     $5$      & 635  & 514  & 11494  & 756      \\
% %     $6$      & 767  & 617  & 13646  & 917      \\
% %     $7$      & 908  & 726  & 16032  & 1076     \\
% %     $8$      & 1038 & 854  & 18421  & 1250     \\
% %     $9$      & 1162 & 945  & 20957  & 1409     \\
% %     $10$     & 1293 & 1045 & 23409  & 1564     \\
%
% # Knuth Morris Pratt
%   \begin{tikzpicture}
%     \begin{semilogyaxis}[
%       xlabel={Benchmark Set},
%       xlabel near ticks,
%       ylabel={Time/ms},
%       legend style = {
%         legend pos=outer north east,
%         cells={anchor=west}
%       },
%       ybar=0pt,
%       ymin=0, ymax=1e4,
%       bar width=4pt,
%       symbolic x coords={16-8,16-64,16-512,32-8,32-64,32-512,64-8,64-64,64-512},
%       xtick=data,
% %       nodes near coords,
% %       nodes near coords align={vertical},
%       x tick label style={rotate=45,anchor=east},
%     ]
%
%       \addplot[color=blue,fill=blue] coordinates {
%         (16-8,   4616)
%         (16-64,  4621)
%         (16-512, 4573)
%         (32-8,   4471)
%         (32-64,  4523)
%         (32-512, 4456)
%         (64-8,   4433)
%         (64-64,  4514)
%         (64-512, 4411)
%       };
%       \addlegendentry{Imperative HOL};
%
%       \addplot[color=red,fill=red] coordinates {
%         (16-8,   499)
%         (16-64,  511)
%         (16-512, 513)
%         (32-8,   453)
%         (32-64,  465)
%         (32-512, 463)
%         (64-8,   418)
%         (64-64,  420)
%         (64-512, 416)
%       };
%       \addlegendentry{C++ (unverified)};
%
%       \addplot[color=green,fill=green] coordinates {
%         (16-8,   597)
%         (16-64,  598)
%         (16-512, 590)
%         (32-8,   551)
%         (32-64,  552)
%         (32-512, 544)
%         (64-8,   530)
%         (64-64,  531)
%         (64-512, 523)
%       };
%       \addlegendentry{LLVM};
%
%
%     \end{semilogyaxis}
%   \end{tikzpicture}
%
%   Execute $a$-$l$ benchmark set from StringBench. Stop at first match.
%
%
% %     16-8     & 499 & 597  & 4616 & 918 \\
% %     16-64    & 511 & 598  & 4621 & 926 \\
% %     16-512   & 513 & 590  & 4573 & 909 \\
% %     32-8     & 453 & 551  & 4471 & 850 \\
% %     32-64    & 465 & 552  & 4523 & 857 \\
% %     32-512   & 463 & 544  & 4456 & 840 \\
% %     64-8     & 418 & 530  & 4433 & 803 \\
% %     64-64    & 420 & 531  & 4514 & 809 \\
% %     64-512   & 416 & 523  & 4411 & 799 \\
%
% \renewcommand{\insertsectitle}{}
%
% # Conclusions
%   * Fast and verified algorithms
%     * LLVM code generator
%     * using Refinement Framework
%     * manageable proof overhead
%   * Case studies
%     * generate really fast, verified code
%     * re-use existing proofs
%   * Current/future work
%     * more complex algorithms
%       * promising (preliminary) results for SAT-solver, Prim's algorithm
%     * deeply embedded semantics
%     * generic Sepref (Imp-HOL, LLVM) $\times$ (nres, nres+time)
%
%   \vfill
%
%   \url{https://github.com/lammich/isabelle_llvm}
%
%
% % #! Malloc
% %   {\small
% %   \begin{minipage}[t]{.46\textwidth}
% % %    \begin{lstlisting}
% % %      allocn (v::val) (s::nat) = do {
% % %        bs <- get;
% % %        set (bs@[Some (replicate s v)]);
% % %        return (ADDR |bs| 0 []) }
% % %    \end{lstlisting}
% %   \end{minipage}
% %   \hfill
% %   \begin{minipage}[t]{.52\textwidth}
% %   \begin{lstlisting}
% %     ll_malloc (s::n word) :: 'a ptr =
% %       assert (n~=0)
% %       allocn (zero_init TYPE('a))
% %   \end{lstlisting}
% %   \end{minipage}
% %   }
%
%
%
%
%
% % # Motivation
% %   *<+-> SAT solvers
% %     * Elaborate optimizations + heuristics
% %     * Highly complex $\implies$ error prone
% %     * Used in safety critical applications
% %   *<+-> Certification
% %     * SAT: Solution. Easy!
% %     * UNSAT: De-facto standard: DRAT
% %   *<+-> DRAT certificate
% %     * Still requires highly optimized checker
% %     * Main work spent on unit propagation (BCP)
% %     * drat-trim is optimized C program
% %       * We found several bugs (now fixed)
% %
% % # DRAT
% %   * Certificate is list of clause additions and deletions
% %   * Added clauses proved redundant
% %     * Must have RUP or RAT property
% %     * Proof based on BCP
% %   * Last clause: Empty clause
% %
% % # Enriched Certificates
% %   Crucial idea
% %     * Unverified \emph{generator} records unit/conflict clauses
% %     * Verified \emph{checker} replays BCP
% %     * \emph{Finding} unit clause replaced by \emph{checking} unit clause
% %   \pause
% %
% %   History
% %     * Cruz-Filipe et al.~[TACAS 2017] for DRUP
% %     * P.~L. [CADE 2017] extended to DRAT
% %     * independently: Cruz-Filipe et al.~[CADE 2017] extended to DRAT
% %
% %
% % # Workflow
% %   \begin{tikzpicture}[node distance = 1.5cm, auto]
% %     \tikzstyle{tool}=[shape=rectangle,draw, minimum size = 5mm]
% %     \tikzstyle{artf}=[minimum size = 5mm]
% %
% %     \node[tool] (sats) {SAT Solver};
% %     \uncover<2->{\node[artf] (drat) [below of = sats] {DRAT certificate};}
% %     \uncover<3->{\node[tool] (gratgen) [below of = drat] {gratgen};}
% %     \uncover<1->{\node[artf] (cnf) [left of = gratgen, anchor=east] {CNF formula};}
% %     \uncover<4->{\node[artf] (grat) [below of = gratgen] {GRAT certificate};}
% %     \uncover<5->{\node[tool] (gratchk) [below of = grat] {gratchk};}
% %     \uncover<6->{\node[artf] (res) [right of = gratchk, anchor=west] {s UNSAT / s ERROR};}
% %
% %
% %     \draw[->] (cnf) |- (sats);
% %     \uncover<2->{\draw[->] (sats) to (drat);}
% %     \uncover<3->{\draw[->] (cnf) to (gratgen);}
% %     \uncover<3->{\draw[->] (drat) to (gratgen);}
% %     \uncover<4->{\draw[->] (gratgen) to (grat);}
% %     \uncover<5->{\draw[->] (grat) to (gratchk);}
% %     \uncover<6->{\draw[->] (gratchk) to (res);}
% %     \uncover<5->{\draw[->] (cnf) |- (gratchk);}
% %
% %
% %     \node<7->[rectangle callout, draw, callout absolute pointer=(gratgen.east), above right = of gratgen, align=left] {Expensive \\ Complex \\ Highly optimized};
% %     \node<8->[rectangle callout, draw, callout absolute pointer=(gratchk.east), above right = of gratchk, align=left] {Cheap \\ Simple \\ Formally verified};
% %
% %   \end{tikzpicture}
% %
% % # Backwards Checking
% %   * Forward phase: Add lemmas, but do no proofs
% %   * Backwards phase: Remove lemma and prove it
% %     * Mark lemmas required for proof
% %     * Skip unmarked lemmas
% %   * Core first heuristics
% %     * Prefer marked lemmas when searching for proof
% %     * Goal: Reduce newly marked lemmas
% %
% % # Novel Optimizations (Compared to drat-trim)
% %   * Multithreaded proving
% %     * Multiple threads in backwards phase
% %     * Synchronize on proved/marked lemmas
% %     * Significantly decreases wall-clock time
% %     * Significantly increases memory usage
% %   * Separate watchlists
% %     * Maintain separate watchlists for marked/unmarked clauses
% %     * Less work in (hot) unit propagation loop
% %     * More work in (cool) analyzer that marks clauses
% %   * RAT-run heuristics
% %     * Re-use gathered RAT candidates for next lemma's proof
% %     * RAT candidates are particular expensive to gather
% %     * Reuse possible in many cases
% %
% % % # Perhaps animation:
% % %   * One thread: Goto next marked lemma, unit-propagation, analyze conflict, mark lemmas
% % %
% % %   * Multiple threads: Goto next marked lemma, try to acquire, up, ana-conf, mark
% %
% %
% % \newcommand{\annot}[1]{{\footnotesize\color{red}{#1}}}
% %
% % \newcommand{\uc}[2]{\uncover<#1->{#2}}
% % # Benchmark Results
% %   \begin{uncoverenv}<1->
% %   Using 2016 SAT competition main track:
% %     * cmsat: 110 UNSAT, 64 SAT  (many RAT lemmas)
% %     * riss6: 128 UNSAT          (Silver medalist, but only RUP)
% %   \end{uncoverenv}
% %
% %   \vfill
% %
% %
% %   \begin{uncoverenv}<2->
% %   \begin{tabular}{c|c|c|c|c|}
% %             & lrat*                             & drat-trim                         & grat/1          & grat/8                    \\\hline
% %       cmsat & \uc{3}{51h \annot{2xT/O, 1xSEGF}} & \uc{4}{42h \annot{2xT/O, 1xSEGF}} & \uc{5}{17h}     & \uc{6}{7h               } \\\hline
% %       riss6 & \uc{3}{42h}                       & \uc{4}{30h                      } & \uc{5}{26h}     & \uc{6}{14h \annot{1xOOM}} \\\hline\hline
% %     $\Sigma$& \uc{3}{93h}                       & \uc{4}{72h                      } & \uc{5}{44h}     & \uc{6}{21h              } \\\hline
% %   \end{tabular}
% %   {\tiny\vspace*{1em}}
% %
% %   {\tiny Wall-clock times, excluded the 3 certificates that drat-trim failed on\\
% %   \begin{tabular}{ll}
% %   {\color{red} T/O} &--- Timeout after 20.000s \\
% %   {\color{red} SEGF} &--- crashed with SIGSEGV \\
% %   {\color{red} OOM} &--- out of memory (128GiB) \\
% %    {*} &--- Incremental version, [Heule et al., ITP'17] \\
% %   \end{tabular}
% %   }
% %   \end{uncoverenv}
% %
% %   \vfill
% %   \begin{uncoverenv}<7->
% %   gratchk time: 12\% (cmsat: 6\%, riss6: 16\%)
% %   \end{uncoverenv}
% %
% %   \vfill
% %
% %   \begin{uncoverenv}<8->
% %   SAT mode: gratchk verified all 64 certificates in 40s
% %   \end{uncoverenv}
% %
% %
% % #![t] Demo
% % %   * Gratgen, gratchk, gratgen/8
% % %   * If possible, on lxcisa0!
% % %
% % %   * Perhaps gratgen no-core-first?
% % %
% % %   * Use rotmul.miter.shuffled-as.sat03-356 ... good to demonstrate RAT-run heuristics, multithreading speedup
% %   \scriptsize
% %   \begin{onlyenv}<1>
% %   \begin{verbatim}
% %   ls -1hs rotmul.miter.shuffled-as.sat03-356.{cnf,drat}
% %   688K rotmul.miter.shuffled-as.sat03-356.cnf
% %    30M rotmul.miter.shuffled-as.sat03-356.drat
% %   \end{verbatim}
% %   \end{onlyenv}
% %
% %   \begin{onlyenv}<2>
% %   \begin{verbatim}
% %   $ drat-trim rotmul.miter.shuffled-as.sat03-356.{cnf,drat}
% %   c parsing input formula with 5980 variables and 35229 clauses
% %   c finished parsing
% %   c detected empty clause; start verification via backward checking
% %   c 30476 of 35229 clauses in core
% %   c 84401 of 219768 lemmas in core using 3834963 resolution steps
% %   c 2162 RAT lemmas in core; 37038 redundant literals in core lemmas
% %   s VERIFIED
% %   c verification time: 13.773 seconds
% %   \end{verbatim}
% %   \end{onlyenv}
% %
% %   \begin{onlyenv}<3>
% %   \begin{verbatim}
% %   $ time ./grat.sh rotmul.miter.shuffled-as.sat03-356
% %   [... output shortened ...]
% %   c Timing statistics (ms)
% %   c Overall:  7246
% %
% %   c Lemma statistics
% %   c RUP lemmas:  79298
% %   c RAT lemmas:  2183
% %   c   RAT run heuristics:   1292
% %
% %   c Reading cnf
% %   c Reading lemmas
% %   c Done
% %   c Verifying unsat (split certificate)
% %   s VERIFIED UNSAT
% %
% %   real    0m1.012s
% %   user    0m0.984s
% %   sys     0m0.028s
% %
% %   real    0m8.268s
% %   user    0m8.168s
% %   sys     0m0.100s
% %
% %   \end{verbatim}
% %   \end{onlyenv}
% %
% %   \begin{onlyenv}<4>
% %   \begin{verbatim}
% %   $ time ./grat.sh rotmul.miter.shuffled-as.sat03-356 -j 3
% %   [... output shortened ...]
% %   c Timing statistics (ms)
% %   c Overall:  4386
% %
% %   c Lemma statistics
% %   c RUP lemmas:  79529
% %   c RAT lemmas:  2169
% %   c   RAT run heuristics:   1269
% %
% %   c Reading cnf
% %   c Reading lemmas
% %   c Done
% %   c Verifying unsat (split certificate)
% %   s VERIFIED UNSAT
% %
% %   real    0m1.011s
% %   user    0m0.988s
% %   sys     0m0.020s
% %
% %   real    0m5.407s
% %   user    0m12.120s
% %   sys     0m0.124s
% %
% %   \end{verbatim}
% %   \end{onlyenv}
% %
% %   \begin{onlyenv}<5>
% %   \begin{verbatim}
% %   $ ls -1hs rotmul.miter.shuffled-as.sat03-356.grat?
% %   5,5M rotmul.miter.shuffled-as.sat03-356.gratl
% %    17M rotmul.miter.shuffled-as.sat03-356.gratp
% %   \end{verbatim}
% %   \end{onlyenv}
% %
% %
% %   \begin{onlyenv}<6>
% %   \begin{verbatim}
% %   $ ls -1hs unif-c500-v250-s1228594393.{cnf,sat}
% %   8,0K unif-c500-v250-s1228594393.cnf
% %   4,0K unif-c500-v250-s1228594393.sat
% %
% %   $ time gratchk sat unif-c500-v250-s1228594393.{cnf,sat}
% %   c Reading cnf
% %   c Reading proof
% %   c Done
% %   c Verifying sat
% %   s VERIFIED SAT
% %
% %   real    0m0.001s
% %   user    0m0.000s
% %   sys     0m0.000s
% %   \end{verbatim}
% %   \end{onlyenv}
% %
% %
% % # Conclusions
% %   Fast SAT solver certification tool
% %   * SAT and UNSAT certificates
% %   * Strong formal correctness guarantees
% %     * formally verified down to int array representing formula
% %   * Available under BSD-style license
% %
% %   \vfill
% %   \center \url{http://www21.in.tum.de/~lammich/grat/}
%
%
% %   Plot:
% %     *# Use rotmul.miter.shuffled-as.sat03-356
% %     * run drat-trim
% %     * run grat toolchain
% %     * run grat toolchain with multiple threads (laptop 3, lxcisa0 8)
% %     * Show sizes of certificates - effective trimming!
%
%
%
%
%
%
%
%
% % % ORIGINAL CADE TALK
% %
% % # SAT
% % %   *<+-> Find solution to cnf formula
% % %     * Widely used, e.g. in hardware verification
% % %     * NP-complete (intractable)
% %   *<+-> SAT solvers
% %     * Elaborate optimizations + heuristics
% %     * Highly complex $\implies$ error prone
% %     * Safety critical
% %   *<+-> Certification
% %     * SAT solver produces certificate
% %     * Checked by simpler program
% %   *<+-> SAT certificate
% %     * Valuation, almost trivial to check
% %   *<+-> UNSAT certificate
% %     * De-facto standard: DRAT
% %     * Demanded by SAT competitions
% %     * Supported by most modern SAT solvers
% %
% % # DRAT-Certificate
% %   *<+-> List of clause additions (lemmas) and deletions
% %   *<+-> Each lemma is \emph{redundant} wrt. current clauses
% %     *<+-> Must have RAT property
% %   *<+-> Last lemma is empty clause
% %
% %   \uncover<+->{
% %   \begin{definition}
% %     Clause $C$ has RAT wrt.\ formula $F$ iff
% %       *{}: $(F \wedge \neg C)^{\textrm u} = \{\square\}$
% %       *or: $\exists l\in C.~\forall D\in F.~\neg l\in D \implies ((F\wedge \neg C)^{\textrm u}\wedge\neg D)^{\textrm u} = \{\square\}$
% %     where $F^{\textrm u}$ means \emph{unit propagation}.
% %   \end{definition}
% %   }
% %
% %
% % # DRAT in Practice
% %   *<+-> Unit propagation and finding RAT candidates
% %     * Computationally expensive
% %   *<+-> Requires highly optimized code and data structures
% %     * Error prone + hard to (formally) verify
% %   *<+-> drat-trim (Heule et al.)
% %     * Standard tool used in SAT competitions
% %     * Written in C, highly optimized, and many features
% %     * Two-watched-literals, backwards checking, core-first, ...
% %     * Sat/Unsat, trimming, unsat core, resolution graph, ...
% % %     * Optimizations: 2wl, bwd checking, core-first, ...
% % %     * Features: Sat/Unsat, trimming, unsat core, resolution graph, ...
% %     *<+-> We found a few bugs (now mostly fixed)
% %       * Erroneous pivot literal check
% %       * Numeric overflows in parser
% %       * Buffer overflows
% %
% %
% % # Story of GRAT
% %   *<+-> I was verifying fast DRAT-checker (in Isabelle)
% %   *<+-> Heard about GRIT [Cruz-Filipe et al., TACAS'17]
% %     * Two-phase checker, but only for DRUP
% %     * First phase: adapted drat-trim tool
% %     * Second phase: verified in Coq
% %     * Both phases: Approx. 2x slower than drat-trim
% %   *<+-> Re-targeted my work to extend this to full DRAT
% %     * Adding multi-threading to first phase
% %     * Using my powerful refinement tools for fast second phase
% %     * My tool was on par with drat-trim, albeit formally verified
% %   *<+-> After CADE-deadline
% %     * Added novel optimizations
% %     * Now significantly faster than drat-trim
% %     * New results included in this talk + proceedings paper
% %   *<+-> Independently: LRAT was developed
% %     * Cruz-Filipe, Heule, et al.
% %
% %
% % # Enriched Certificates
% %   \begin{onlyenv}<1>
% %   \begin{tikzpicture}[node distance = 1.5cm, auto]
% %     \tikzstyle{tool}=[shape=rectangle,draw, minimum size = 5mm]
% %     \tikzstyle{artf}=[minimum size = 5mm]
% %
% %     \node[artf] (drat) {DRAT certificate};
% %     \node[tool] (drat-trim) [below of = drat] {drat-trim};
% %     \node[artf] (cnf) [left of = drat-trim, anchor=east] {CNF formula};
% %     \node[artf] (res) [right of = drat-trim, anchor=west] {s UNSAT / s ERROR};
% %
% %     \draw[->] (cnf) to (drat-trim);
% %     \draw[->] (drat) to (drat-trim);
% %     \draw[->] (drat-trim) to (res);
% %   \end{tikzpicture}
% %   \end{onlyenv}
% %
% %   \begin{onlyenv}<2->
% %   \begin{tikzpicture}[node distance = 1.5cm, auto]
% %     \tikzstyle{tool}=[shape=rectangle,draw, minimum size = 5mm]
% %     \tikzstyle{artf}=[minimum size = 5mm]
% %
% %     \node[artf] (drat) {DRAT certificate};
% %     \node[tool] (gratgen) [below of = drat] {gratgen};
% %     \node[artf] (cnf) [left of = gratgen, anchor=east] {CNF formula};
% %     \node[artf] (grat) [below of = gratgen] {GRAT certificate};
% %     \node[tool] (gratchk) [below of = grat] {gratchk};
% %     \node[artf] (res) [right of = gratchk, anchor=west] {s UNSAT / s ERROR};
% %
% %
% %     \draw[->] (cnf) to (gratgen);
% %     \draw[->] (drat) to (gratgen);
% %     \draw[->] (gratgen) to (grat);
% %     \draw[->] (grat) to (gratchk);
% %     \draw[->] (gratchk) to (res);
% %     \draw[->] (cnf) |- (gratchk);
% %
% %     \begin{uncoverenv}<3->
% %     \node[rectangle callout, draw, callout absolute pointer=(gratgen.east), above right = of gratgen, align=left] {Expensive \\ Complex \\ Highly optimized};
% %     \end{uncoverenv}
% %     \node<4->[rectangle callout, draw, callout absolute pointer=(gratchk.east), above right = of gratchk, align=left] {Cheap \\ Simple \\ Formally verified};
% %
% %   \end{tikzpicture}
% %   \end{onlyenv}
% %
% % # Moving Complexity to gratgen
% %   *<+-> Checking direction
% %     * gratgen works in backwards mode, trims certificate
% %     * gratchk works in forward mode
% %   *<+-> Unit propagation
% %     * gratgen outputs unit/conflict clauses
% %     * gratchk only checks that clauses are unit/conflict
% %     * Cruz-Filipe et al. [TACAS'17]
% %   *<+-> RAT candidate counts
% %     * gratgen gathers RAT candidates, and outputs RAT counts
% %     * gratchk only checks that candidates are exhaustive
% %       * maintaining candidate lists based on RAT counts
% %
% % # gratgen
% %   *<+-> Initially: Clone of drat-trim's backwards mode
% %     * Uses two-watched literals, core-first unit propagation
% %     * Implemented in \CC11
% %   *<+->+: Separate watchlists: Faster core-first propagation
% %   *<+->+: RAT-run heuristics: Faster RAT-candidate finding
% %   *<+->+: Optional multithreading: Parallel checking of lemmas
% %
% % # gratchk
% %   *<+-> Streaming for lower memory footprint
% %     * Unit/conflict clauses and deletion information not kept in memory
% %   *<+-> Formally verified with Isabelle/HOL
% % %     * LCF-style prover, reducing trust to small logical inference kernel
% % %   * Uses imperative data structures
% % %     * quite fast for verified program!
% %   *<+-> Verification uses stepwise refinement approach
% %
% % # Stepwise Refinement
% %   *<+-> Abstract description of algorithm
% %     * Easy to verify
% %   *<+-> Refine towards implementation
% %     * Small, independent, correctness preserving steps
% %   *<+-> Goal: Modularization of proofs
% %     * Efficient implementation does not affect abstract correctness proof
% %     * Making big proofs manageable in first place
% %   *<+-> Isabelle Refinement Framework
% %     * Powerful tool support for refinement proofs
% %     * Library of verified (imperative) data structures
% %
% % #! Refinement of gratchk --- Checker State
% %   \begin{uncoverenv}<1->
% %   Abstract: Using Sets and Functions
% %   \begin{lstlisting}
% %     literal = Pos var | Neg var
% %     clause = literal set
% %     clausemap = (id => clause option) \<times> (literal => id set option)
% %     state = id \<times> clausemap \<times> (var => bool option)
% %   \end{lstlisting}
% %   \end{uncoverenv}
% %
% %   \begin{uncoverenv}<2->
% %   Refinement \#1: Using iterators for clauses, lists for RAT candidates
% %   \begin{lstlisting}
% %     clausemap1 = (id => $\color{red}\textsf{it}$) \<times> (literal => id $\color{red}\textsf{list}$ option)
% %     state1 = id \<times> clausemap1 \<times> (var => bool option)
% %   \end{lstlisting}
% %   \end{uncoverenv}
% %
% %   \begin{uncoverenv}<3->
% %   Refinement \#2: Iterators as indexes into array, literals as integers, functions as arrays
% %   \begin{lstlisting}
% %     clausedb2 = $\color{red}\textsf{int array}$
% %     clausemap2 = $\color{red}\textsf{nat array}$ \<times> id list option $\color{red}\textsf{array}$
% %     state2 = id \<times> clausemap2 \<times> $\color{red}\textsf{nat array}$
% %   \end{lstlisting}
% %   \end{uncoverenv}
% %
% % #! Final Correctness Statement
% %   \begin{uncoverenv}<+->
% %   \begin{lstlisting}
% %       <DBi |->$_a$ DB>
% %         verify_unsat_impl DBi F_end it prf
% %       <\<lambda>r. DBi |->$_a$ DB * \<up>(r ==> /@formula_unsat_spec@/ DB F_end)>
% %   \end{lstlisting}
% %   \end{uncoverenv}
% %
% %   \begin{uncoverenv}<+->
% %   where
% %   \begin{lstlisting}
% %     /@formula_unsat_spec@/ :: "int list => nat => bool"
% %     /@formula_unsat_spec@/ DB F_end == (
% %       let lst = tl (take F_end DB) in
% %         1 < F_end \<and> F_end \<le> length DB \<and> last lst = 0
% %         \<and> (\<nexists>\<sigma>. /@assn_consistent@/ \<sigma> \<and> (\<forall>C\<in>set (tokenize 0 lst). \<exists>l\<in>set C. \<sigma> l)))
% %   \end{lstlisting}
% %   \end{uncoverenv}
% %
% %   \begin{uncoverenv}<+->
% %   where
% %   \begin{lstlisting}
% %     /@assn_consistent@/ :: "(int => bool) => bool"
% %     "/@assn_consistent@/ \<sigma> = (\<forall>x. x\<noteq>0 ==> \<not> \<sigma> (-x) = \<sigma> x)"
% %   \end{lstlisting}
% %   \end{uncoverenv}
% %
% %   \begin{uncoverenv}<+->
% %   Trusted base: Hoare triples + standard list functions
% %   \end{uncoverenv}
% %
% % #! Generating gratchk
% %   *<+-> Export from Isabelle to Standard/ML
% %     * Trusted: Isabelle code generator (extended pretty printing)
% %   *<+-> Add parser and command line interface
% %     * Trusted: Parsing DIMACS to int array (<100 LOC, simple)
% %       * Standard ML detects overflows by design
% %     * Trusted: CLI (straightforward)
% %   *<+-> Compile with MLton
% %     * Trusted: Compiler
% %
% % \newcommand{\annot}[1]{{\footnotesize\color{red}{#1}}}
% %
% % \newcommand{\uc}[2]{\uncover<#1->{#2}}
% %
% %
% % # Benchmark Results
% %   \begin{uncoverenv}<1->
% %   Using 2016 SAT competition main track:
% %     * cmsat: 110 UNSAT, 64 SAT  (many RAT lemmas)
% %     * riss6: 128 UNSAT          (Silver medalist, but only RUP)
% %   \end{uncoverenv}
% %
% %   \vfill
% %
% %
% %   \begin{uncoverenv}<2->
% %   \begin{tabular}{c|c|c|c|c|}
% %             & lrat*                             & drat-trim                         & grat/1          & grat/8                    \\\hline
% %       cmsat & \uc{3}{51h \annot{2xT/O, 1xSEGF}} & \uc{4}{42h \annot{2xT/O, 1xSEGF}} & \uc{5}{17h}     & \uc{6}{7h               } \\\hline
% %       riss6 & \uc{3}{42h}                       & \uc{4}{30h                      } & \uc{5}{26h}     & \uc{6}{14h \annot{1xOOM}} \\\hline\hline
% %     $\Sigma$& \uc{3}{93h}                       & \uc{4}{72h                      } & \uc{5}{44h}     & \uc{6}{21h              } \\\hline
% %   \end{tabular}
% %   {\tiny\vspace*{1em}}
% %
% %   {\tiny Wall-clock times, excluded the 3 certificates that drat-trim failed on\\
% %   \begin{tabular}{ll}
% %   {\color{red} T/O} &--- Timeout after 20.000s \\
% %   {\color{red} SEGF} &--- crashed with SIGSEGV \\
% %   {\color{red} OOM} &--- out of memory (128GiB) \\
% %    {*} &--- Incremental version, [Heule et al., ITP'17] \\
% %   \end{tabular}
% %   }
% %   \end{uncoverenv}
% %
% %   \vfill
% %   \begin{uncoverenv}<7->
% %   gratchk time: 12\% (cmsat: 6\%, riss6: 16\%)
% %   \end{uncoverenv}
% %
% %   \vfill
% %
% %   \begin{uncoverenv}<8->
% %   SAT mode: cmsat: 40s
% %   \end{uncoverenv}
% %
% % # Conclusions
% %   Fast SAT solver certification tool
% %   * SAT and UNSAT certificates
% %   * Strong formal correctness guarantees
% %     * formally verified down to int array representing formula
% %   * Available under BSD-style license
% %
% %   \vfill
% %   \center \url{http://www21.in.tum.de/~lammich/grat/}
% %
% %   \vfill
% %   Tool Paper on SAT'2017
% %
% %
% %
% % %
% % % # Enriched Certificates
% % %   * Idea: Replace unit-propagation by unit checking
% % %     * Cruz-Filipe et al. [TACAS'17]: For (weaker) DRUP certificates
% % %   * During certificate checking: $(F \wedge \neg C)^{\textrm u} = \lightning$
% % %     * Output list of clauses in order they become unit
% % %     * Output conflict clause
% % %     * Trim certificate
% % %   * Then run 2nd checker
% % %     * Against original formula, using enriched certificate
% % %     * Only check that clauses are actually unit/conflict
% % %     * Simple + cheap $\implies$ amenable to formal verification
% % %
% %
    
\end{document}




