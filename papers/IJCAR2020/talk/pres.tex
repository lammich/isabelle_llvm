\documentclass[fleqn]{beamer}


\mode<presentation>
{
  \usetheme{default}
%   \useinnertheme[shadow=true]{rounded}

  \useinnertheme{circles}
  
%    \useoutertheme{infolines}
  % or ...
  \setbeamersize{text margin left=1em,text margin right=1em}

%   \setbeamercovered{transparent}
  % or whatever (possibly just delete it)

  \beamertemplatenavigationsymbolsempty
  
% Display frame numbers in footline
  \setbeamertemplate{footline}[frame number]
}

\usepackage{etex}
\usepackage[utf8]{inputenc}
\usepackage[T1]{fontenc}
\usepackage[english]{babel}
\usepackage{amsmath}
% \usepackage{amsthm}
% \usepackage{stmaryrd}
\usepackage{lmodern}
\usepackage{mathpartir}

\usepackage{centernot}

\usepackage{colortbl}
\usepackage{multirow}

\usepackage[purexy]{qsymbols}
\usepackage{graphicx}
\usepackage[nobeamer]{graphbox}

\usepackage{cprotect}
\usepackage{listings}
\usepackage{lstautogobble}
\usepackage{relsize}

\usepackage[noend]{algpseudocode}

\usepackage{ifthen}
\usepackage{pgf}
\usepackage{tikz}
\usetikzlibrary{scopes}
\usetikzlibrary{decorations}
  \usetikzlibrary{decorations.pathmorphing}
\usetikzlibrary{arrows}
\usetikzlibrary{automata}
\usetikzlibrary{positioning}
\usetikzlibrary{chains}
\usetikzlibrary{shapes.geometric}
\usetikzlibrary{shapes.callouts}
\usetikzlibrary{shapes.misc}
\usetikzlibrary{shapes.multipart}
\usetikzlibrary{fit}
\usetikzlibrary{calc}

\usepackage{pgfplots}

% \tikzstyle{stack}=[inner sep=0pt,minimum size=2mm]
% \tikzstyle{ssline}=[->,snake=snake,segment amplitude=.2mm,segment length=3mm,line after snake=1mm]
% \tikzstyle{fgnode}=[circle,draw,inner sep=0pt,minimum size=2mm]

\tikzset{
  -|-/.style={
    to path={
      (\tikztostart) -| ($(\tikztostart)!#1!(\tikztotarget)$) |- (\tikztotarget)
      \tikztonodes
    }
  },
  -|-/.default=0.5,
  |-|/.style={
    to path={
      (\tikztostart) |- ($(\tikztostart)!#1!(\tikztotarget)$) -| (\tikztotarget)
      \tikztonodes
    }
  },
  |-|/.default=0.5,
}


\tikzset{
  invisible/.style={opacity=0},
  visible on/.style={alt=#1{}{invisible}},
  alt/.code args={<#1>#2#3}{%
    \alt<#1>{\pgfkeysalso{#2}}{\pgfkeysalso{#3}} % \pgfkeysalso doesn't change the path
  },
}



% The goal is to translate
%    \overlaynode<red,blue>{hallo};
% into
%    \node[red]{hallo};
%    \node[blue]{hallo};
\makeatletter
\def\overlaynode<#1>#2;{
        \gdef\stacknodecommonpart{#2}
        \pgfkeys{/typeset node/.list={#1}}
        % we are lazy
        % pgfkeys will translate
        %    \pgfkeys{/typeset node/.list={red,blue}}
        % into
        %    \pgfkeys{/typeset node=red}
        %    \pgfkeys{/typeset node=blue}
}
\pgfkeys{
    /typeset node/.code={
        \edef\pgf@marshal{\noexpand\node[#1]\stacknodecommonpart;}
        \pgf@marshal
    }
}
% \tikz{
%     \overlaynode<red,{blue,xshift=1}>{{Hello}}; % notice the nested {{}}
% %   \overlaynode<red,{blue,xshift=1}>[]{Hello}; % workaround
%     \overlaynode<red,{blue,rotate=5}>at(1,0)[draw]{from};
%     \overlaynode<red,{blue,scale=1.1}>[circle]at(2,0)[draw]{the};
%     \overlaynode<red,{blue,opacity=.5}>[fill=yellow!50]at(3,0){other};
%     \overlaynode<{red,rectangle callout,fill},{blue,ellipse callout,draw}>at(4,0){side};
% }
% \tikz{
%     \overlaynode<
%             {fill=red,callout absolute pointer=(45:2)},
%             {fill=yellow,callout absolute pointer=(135:2),callout pointer shorten=1cm},
%             {fill=green,callout relative pointer=(-135:2),callout pointer width=.5cm},
%             {fill=blue,callout relative pointer=(-45:2),callout pointer shorten=1cm,callout pointer width=.5cm}
%         >
%         [rectangle callout,text=white]
%         at(0,0){can you}
%     ;
% }

\def\overlaynodedrawfill{\pgfutil@ifnextchar[{\overlaynodedrawfill@opt}{\overlaynodedrawfill@opt[]}}
\def\overlaynodedrawfill@opt[#1]<#2>#3;{
    \begin{scope}[transparency group,draw=black,fill=white,line cap=round,line join=round,#1]
        \pgfmathsetmacro\pgflinewidthdouble{2\pgflinewidth}
        \overlaynode<#2>[draw=pgfstrokecolor,line width=\pgflinewidthdouble]#3;
        \overlaynode<#2>[fill=pgffillcolor]#3;
    \end{scope}
}

\makeatother


% \tikz{
%     \overlaynodedrawfill[draw=magenta,fill=cyan,opacity=.5]<
%             {callout absolute pointer=(45:2)},
%             {callout absolute pointer=(135:2),callout pointer shorten=1cm},
%             {callout relative pointer=(-135:2),callout pointer width=.5cm},
%             {callout relative pointer=(-45:2),callout pointer shorten=1cm,callout pointer width=.5cm}
%         >
%         [rectangle callout]
%         at(0,0){hear me}
%     ;
% }
%
% \tikz{
%     \overlaynodedrawfill<
%             {cloud callout,callout absolute pointer=(90:2),inner sep=-20},
%             {rectangle callout,callout absolute pointer=(210:2),minimum height=30},
%             {ellipse callout,callout absolute  pointer=(-30:2)}
%         >
%         at(0,0){nice to meet you}
%     ;
% }
%
% \tikz{
%     \overlaynodedrawfill<
%             {starburst},
%             {cloud,inner sep=-10}
%         >
%         at(0,0){where you been}
%     ;
% }
%
% \usetikzlibrary{shapes.arrows}
% \tikz{
%     \overlaynodedrawfill[arrow box,text opacity=0,minimum height=50,minimum width=50,inner xsep=-10]<
%             {rotate=22.5},
%             {rotate=45},
%             {rotate=67.5},
%             {text opacity=1}
%         >
%         at(0,0){I can}
%     ;
% }
%








\usepackage{packages/isabelle}
\usepackage{packages/isabelletags}
\usepackage{packages/isabellesym}
\usepackage{packages/comment}

% \isabellestyle{it}

\def\isachardoublequote{}%
\def\isachardoublequoteopen{}%
\def\isachardoublequoteclose{}%

\newcommand{\isainnerkeyword}[1]{{\textbf{#1}}}
\newcommand{\isasymexistsA}{\isamath{\exists_{\textsc A}\,}}


\def\isadelimproof{}
\def\endisadelimproof{}
\def\isatagproof{}
\def\endisatagproof{}
\def\isafoldproof{}
\def\isadelimproof{}
\def\endisadelimproof{}

\def\isastylescript{\sl}%


% Not really meant for highlighting isabelle source, but for easily writing latex that looks like
% isabelle
% 
% keyword level 1 - isabelle outer syntax 
% keyword level 2 - isabelle inner syntax programming constructs (if, let, etc)
% keyword level 3 - standard constants (length, mod, etc)
% keyword level 4 - isabelle proof methods


% \newcommand{\lsem}{\ensuremath{\mathopen{[\![}}}
% \newcommand{\rsem}{\ensuremath{\mathclose{]\!]}}}

\lstdefinelanguage{isabelle}{
  morekeywords={theorem,theorems,corollary,lemma,lemmas,locale,begin,end,fixes,assumes,shows,and,class,
    constrains , definition, where, apply, done,unfolding, primrec, fun, using, by, for, uses,
    schematic_lemma, concrete_definition, prepare_code_thms, export_code, datatype, type_synonym, typedef, value,
    proof, next, qed, show, have, hence, thus, interpretation, fix, context, sepref_definition,is
 } ,
  morekeywords=[2]{rec, return, bind, foreach, if, then, else, do, let, in, res, spec, fail, assert, while, case, of,
    check},
%  morekeywords=[3]{length,mod,insert},
%   morekeywords=[4]{simp,auto,intro,elim,rprems,refine_mono,refine_rcg},
  sensitive=True,
  morecomment=[s]{(\*}{\*)},
}

\lstset{
    language=isabelle,
    mathescape=true,
    escapeinside={--"}{"},
    basicstyle={\itshape},
    keywordstyle=\rm\bfseries,
    keywordstyle=[2]\rm\tt,
    keywordstyle=[3]\rm,
    keywordstyle=[4]\rm,
    showstringspaces=false,
    keepspaces=true,
    columns=[c]fullflexible}
\lstset{literate=
  {"}{}0
  {'}{{${}^\prime\!$}}1
  {\%}{{$\lambda$}}1
  {\\\%}{{$\lambda$}}1
  {\\\$}{{$\mathbin{\,\$\,}$}}1
  {->}{{$\rightarrow$}}1
  {<-}{{$\leftarrow$}}1
  {<.}{{$\langle$}}1
  {.>}{{$\rangle$}}1
  {<=}{{$\le$}}1
  {>=}{{$\ge$}}1
  {<->}{{$\leftrightarrow$}}1
  {-->}{{$\longrightarrow$}}2
  {<-->}{{$\longleftrightarrow$}}1
  {=>}{{$\Rightarrow$}}1
  {==}{{$\equiv$}}2
  {==>}{{$\implies$}}2
  {<=>}{{$\Leftrightarrow$}}1
  {~=}{{$\ne$}}1
  {|}{{$\mid$}}1
  {-`}{{$\rightharpoonup$}}1
  {|`}{{$\restriction$}}1
  {!!}{{$\bigwedge$}}1
  {(}{{$($}}1
  {)}{{$)$}}1
  {\{}{{$\{$}}1
  {\}}{{$\}$}}1
  {[}{{$[$}}1
  {]}{{$]$}}1
  {[|}{{$\llbracket$}}1
  {|]}{{$\rrbracket$}}1
  {\\<lbrakk>}{{$\lsem$}}1
  {\\<rbrakk>}{{$\rsem$}}1
  {|-}{{$\vdash$}}1
  {|=}{{$\models$}}1
  {|->}{{$\mapsto$}}1
  {|_|}{{$\bigsqcup$}}1
  {...}{{$\dots$}}1
  {\\x}{{$\times$}}1
  {_0}{{${}_0$}}1
  {_1}{{${}_1$}}1
  {_2}{{${}_2$}}1
  {_3}{{${}_3$}}1
  {_4}{{${}_4$}}1
  {_5}{{${}_5$}}1
  {_6}{{${}_6$}}1
  {_7}{{${}_7$}}1
  {_8}{{${}_8$}}1
  {_9}{{${}_9$}}1
  {_L}{{${}_L$}}1
  {\\_n}{{${}_n$}}1
  {\\_i}{{${}_i$}}1
  {\\_j}{{${}_j$}}1
  {\\_x}{{${}_x$}}1
  {\\_y}{{${}_y$}}1
  {\\impl}{{${}_\dagger$}}1
  {^*}{{$^*$}}1
  {^k}{{$^k$}}1
  {^d}{{$^d$}}1
  {\\<^sup>*}{{$^*$}}1
  {\\<^sub>*}{{$_*$}}1
  {\\<^sub>A}{{$_A$}}1
  {\\<^sub>r}{{$_r$}}1
  {\\<^sub>a}{{$_a$}}1
  {:_i}{{$:_i$}}1
  {\\<A>}{{$\mathcal{A}$}}1
  {\\<O>}{{\sf o}}1
  {\\<Phi>}{{$\Phi$}}1
  {\\<Psi>}{{$\Psi$}}1
  {\\<sigma>}{{$\sigma$}}1
  {\\<cdot>}{{$\cdot$}}1
  {\\<in>}{{$\in$}}1
  {\\<le>}{{$\le$}}1
  {\\<noteq>}{{$\ne$}}1
  {\\<lambda>}{{$\lambda$}}1
  {\\<longrightarrow>}{{$\longrightarrow$}}1
  {\\<longleftrightarrow>}{{$\longleftrightarrow$}}1
  {\\<Rightarrow>}{{$\Rightarrow$}}1
  {\\<Longrightarrow>}{{$\Longrightarrow$}}1
  {\\<rightarrow>}{{$\rightarrow$}}1
  {\\<leftarrow>}{{$\leftarrow$}}1
  {\\<mapsto>}{{$\mapsto$}}1
  {\\<equiv>}{{$\equiv$}}1
  {\\<and>}{{$\and$}}1
  {\\<or>}{{$\vee$}}1
  {\\<And>}{{$\bigwedge$}}1
  {\\<Up>}{{$\Uparrow$}}1
  {\\<Down>}{{$\Downarrow$}}1
  {\\<Union>}{{$\bigcup$}}1
  {\\<up>}{{$\uparrow$}}1
  {\\<down>}{{$\downarrow$}}1
  {\\<times>}{{$\times$}}1
  {\\<forall>}{{$\forall$}}1
  {\\<exists>}{{$\exists$}}1
  {\\<nexists>}{{$\nexists$}}1
  {\\<union>}{{$\cup$}}1
  {\\<inter>}{{$\cap$}}1
  {\\<subset>}{{$\subset$}}1
  {\\<subseteq>}{{$\subseteq$}}1
  {\\<supset>}{{$\supset$}}1
  {\\<supseteq>}{{$\supseteq$}}1
  {\\<alpha>}{{$\alpha$}}1
  {\\<beta>}{{$\beta$}}1
  {\\<gamma>}{{$\gamma$}}1
  {\\alpha}{{$\alpha$}}1
  {\\beta}{{$\beta$}}1
  {\\gamma}{{$\gamma$}}1
  {\\<Gamma>}{{$\Gamma$}}1
  {\\<langle>}{{$\langle$}}1
  {\\<rangle>}{{$\rangle$}}1
  {\\<not>}{{$\neg$}}1
  {\\<box>}{{$\oblong$}}1
  {\\<bot>}{{$\bot$}}1
  {\\<top>}{{$\top$}}1
  {\\<notin>}{{$\notin$}}1
  {\\<guillemotright>}{{$\gg$}}1
  {\\in}{$\in$}1
  {\\and}{$\wedge$}1
  {\\or}{$\vee$}1
  {\\Phi}{{$\Phi$}}1
  {\\Psi}{{$\Psi$}}1
  {\\le}{{$\le$}}1
  {\\Up}{{$\Uparrow$}}1
  {\\Down}{{$\Down$}}1
  {>>}{{$\gg$}}1
  {>>=}{{${\gg}{=}$}}1
  {<*lex*>}{{$\times_{\sf lex}$}}1
  {\\<open>}{{\rm\guilsinglleft}}1
  {\\<close>}{{\rm\guilsinglright}}1
}

% \newcommand{\is}{\lstinline[language=isabelle,basicstyle=\normalsize\ttfamily\slshape]}
\newcommand{\is}{\lstinline[language=isabelle]}
\newcommand{\q}[1]{\mbox{\guilsinglleft{#1}\hspace{-2.5pt}\guilsinglright}}
% \newcommand{\isai}[1]{\q{\lstinline[language=isabelle,basicstyle=\normalsize\ttfamily\slshape]{#1}}}
\cMakeRobust\q

\newcommand{\isai}{\lstinline[language=isabelle,basicstyle=\normalsize\ttfamily\slshape]}

\newcommand\CC{C\nolinebreak[4]\hspace{-.05em}\raisebox{.3ex}{\relsize{-1}{\textbf{++}}}}

\newcommand{\eqdef}{\mathrel{{=}_{def}}}
\newcommand{\iffdef}{\mathrel{{\mathord{\iff}\!\!}_{def}}}


\makeatletter
\newcommand*{\overlaynumber}{\number\beamer@slideinframe}
\makeatother

\AtBeginSection[] % Do nothing for \section*
{
  \begin{frame}<beamer>
    \frametitle{Outline}
    \tableofcontents[currentsection]
  \end{frame}
}


\title{Efficient Verified Implementation of Introsort and Pdqsort}

% \subtitle
% {Subtitle} % (optional)

\author[Peter Lammich]{Peter Lammich}
% - Use the \inst{?} command only if the authors have different
%   affiliation.

\institute[UoM] % (optional, but mostly needed)
{ The University of Manchester}
% - Use the \inst command only if there are several affiliations.
% - Keep it simple, no one is interested in your street address.

\date {July 2020}
% {2008-12-01}


% If you have a file called "university-logo-filename.xxx", where xxx
% is a graphic format that can be processed by latex or pdflatex,
% resp., then you can add a logo as follows:

% \pgfdeclareimage[height=0.5cm]{university-logo}{university-logo-filename}
% \logo{\pgfuseimage{university-logo}}


% Delete this, if you do not want the table of contents to pop up at
% the beginning of each subsection:


% If you wish to uncover everything in a step-wise fashion, uncomment
% the following command: 

%\beamerdefaultoverlayspecification{<+->}

%\mathchardef\-="2D
%\renewcommand\-{\text{-}}

\newcommand{\mc}{\color{blue}}
\newcommand{\term}[1]{{\mc#1}}

\let\olddisplaystyle\displaystyle
\newcommand{\mydisplaystyle}{\olddisplaystyle\mc}
\let\displaystyle\mydisplaystyle

\newcommand{\smc}{\everymath{\mc}}
\smc

\lstset{basicstyle=\color{blue}}

% \newcommand<>{\btikzset}[2]{\alt#3{\tikzset{#1}}{\tikzset{#2}}}

\tikzset{onslide/.code args={<#1>#2}{%
  \only<#1>{\pgfkeysalso{#2}} % \pgfkeysalso doesn't change the path
}}

\tikzset{uncover/.code args={<#1>#2}{%
  \uncover<#1>{\pgfkeysalso{#2}} % \pgfkeysalso doesn't change the path
}}


\tikzset{>=latex}


\lstset{autogobble}

\newcommand{\natN}{{\text{nat}_{\mathord{<}N}}}

\newcommand{\high}[1]{{\color{blue}#1}}


\newcommand{\false}{\textrm{false}}
\newcommand{\true}{\textrm{true}}

\let\oldCall\Call

\renewcommand{\Call}[2]{\oldCall{\color{black}#1}{#2}}


\begin{document}
% \input{macros}

\begin{frame}
  \titlepage
\end{frame}


\newcommand{\insertsectitle}{}

\setbeamertemplate{frametitle}{\vspace{.7em}\insertframetitle\hfill \small\raisebox{10pt}{\insertsectitle}}

\begin{frame}{Motivation + Overview}
  \begin{itemize}
   \item<+-> Verification of efficient software
    \begin{itemize}
     \item stepwise refinement: separation of concerns
      \begin{itemize}
       \item algorithmic idea, data structures, optimizations, ...
      \end{itemize}
     \item interactive theorem prover: flexible, mature
      \begin{itemize}
       \item easily proves required background theory
      \end{itemize}
    \end{itemize}
   \item<+-> Isabelle Refinement Framework
    \begin{itemize}
     \item tools for stepwise refinement in Isabelle/HOL
     \item already used for complex software: Model-Checkers, UNSAT-Certifiers, Graph Algorithms, ...
    \end{itemize}
   \item<+-> Backends (refinement target)
    \begin{itemize}
     \item limited by Isabelle's code generator
     \item purely functional code: {\color{red}slow}
     \item functional + imperative (e.g. Standard ML): {\color{orange}faster}
     \item<+-> cannot compete with good C/C++ compiler!
%     * purely imperative (LLVM): {\color{green}super-fast}

    \end{itemize}
  \end{itemize}
\end{frame}
{
\usebackgroundtemplate{\tikz\node[opacity=0.2] {\includegraphics[height=\paperheight]{isabelle-llvm.png}};}

\begin{frame}{Isabelle-LLVM}
%   \center{\includegraphics[width=3cm]{isabelle-llvm.png}}
  \begin{itemize}
   \item<+-> Fragment of LLVM semantics formalized in Isabelle/HOL
    \begin{itemize}
     \item code generator for LLVM code and C/C++ headers
     \item integration with Isabelle Refinement Framework
     \item slim trusted code base (vs.~functional lang.~compiler)
    \end{itemize}
   \item<+-> {\color{blue}Can now compete with C/C++ implementations}
    \begin{itemize}
     \item less features (datatype, poly, ...) require more complex refinement
     \item higher-level refinements can typically be reused

    \end{itemize}
  \end{itemize}
\end{frame}
}


\begin{frame}{This Paper: Overview}
  Case study how fast we can get
  \onslide<+->

  \begin{itemize}
   \item<+-> Verify state-of-the-art generic sorting algorithms
    \begin{itemize}
     \item Introsort (std::sort in libstdc++)
     \item Pdqsort (from Boost C++ Libraries)
    \end{itemize}
   \item<+-> Using Isabelle Refinement Framework
    \begin{itemize}
     \item separate optimizations from algorithmic ideas
     \item usable as building-blocks for other verifications
    \end{itemize}
   \item<+-> As fast as their unverified counterparts
    \begin{itemize}
     \item on an extensive set of benchmarks

    \end{itemize}
  \end{itemize}
\end{frame}
\begin{frame}{The Introsort Algorithm}
  \begin{itemize}
   \item Combine quicksort, heapsort, and insort to fast $O(n\log n)$ algorithm.
    \begin{itemize}
     \item<2-> if quicksort recursion too deep, switch to heapsort
     \item<3-> use insertion sort for small partitions
      \begin{itemize}
       \item final insort on array sorted up to threshold

      \end{itemize}
    \end{itemize}
  \end{itemize}
  \newcommand{\HiLi}{\raisebox{.5ex}{\tikz{\draw[<-,thick,red] (0,0) -- (1,0);}}}

  \begin{algorithmic}[1]
    \Procedure{introsort}{$xs,l,h$}
      \If{$h-l>1$} \label{l:intrs:trivial}
        \State{\Call{introsort\_aux}{$xs,l,h,2\lfloor\log_2(h-l)\rfloor$}}
        \State{\Call{final\_insort}{$xs,l,h$}} \only<3>{\HiLi}
      \EndIf
    \EndProcedure

    \Procedure{introsort\_aux}{$xs,l,h,d$}
      \If{$h-l>\textrm{threshold}$} \only<3>{\HiLi}
        \If{$d=0$}~\Call{heapsort}{$xs,l,h$} \only<2>{\HiLi}
        \Else
          \State{$m\gets\Call{partition\_pivot}{xs,l,h}$}
          \State{\Call{introsort\_aux}{$xs,l,m,d-1$}}
          \State{\Call{introsort\_aux}{$xs,m,h,d-1$}}
        \EndIf
      \EndIf
    \EndProcedure
  \end{algorithmic}
    
\end{frame}
\begin{frame}[fragile]{Verification Methodology: Modularity}
  \begin{itemize}
   \item Specifications for subroutines, e.g. \isai{heapsort <= sort_spec}
    \begin{itemize}
     \item proof only uses specification
     \item independant of impl details of subroutines

    \end{itemize}
  \end{itemize}
  \pause
  \begin{lstlisting}
    partition_spec xs == $\text{\color{gray} --- any non-trivial partitioning}$
      assert (length xs >= 4);
      spec (xs_1,xs_2). mset xs = mset xs_1 + mset xs_2 \<and> xs_1~=[] \<and> xs_2~=[]
            \<and> (\<forall>x\<in>set xs. \<forall>y\<in>set ys. x <= y)

  \end{lstlisting}
  \pause
  \begin{lstlisting}
    part_sorted_spec xs == $\text{\color{gray} --- sort up to threshold}$
      spec xs'. mset xs' = mset xs \<and> part_sorted_wrt (<=) threshold xs'

    $\text{\color{gray} where}$
      part_sorted_wrt n xs == \<exists>ss. is_slicing n xs ss \<and> sorted_wrt slice_lt ss"
      is_slicing n xs ss == xs = concat ss \<and> (\<forall>s\<in>set ss. s~=[] \<and> length s <= n)"
      slice_lt xs ys == \<forall>x\<in>set xs. \<forall>y\<in>set ys. x <= y

  \end{lstlisting}

\end{frame}
\begin{frame}[fragile]{Verification Methodology: Refinement}
  \begin{itemize}
   \item<+-> E.g. lists $\rightarrow$ slices of lists $\rightarrow$ arrays; $\mathbb N$ $\rightarrow$ uint64\_t
    \onslide<+->
    \begin{lstlisting}
      introsort_aux1 d xs <= part_sorted_spec xs $\text{\color{gray} --- sort whole list}$

      (xsi,xs)\<in>slicep_rel l h ==> $\text{\color{gray} --- sort slice}$
        introsort_aux2 d xsi l h <= \<Down>(slice_rel xsi l h) (introsort_aux1 d xs)

      (introsort_aux_impl, introsort_aux2) $\text{\color{gray} --- sort arrays, indices as uint64}$
        : nat$_{64}$ -> array^d -> nat$_{64}$ -> nat$_{64}$ -> array
    \end{lstlisting}

   \item<+-> Transitivity yields
    \begin{lstlisting}
      (introsort_aux_impl, %d. slice_part_sorted_spec)
        : nat$_{64}$ -> array^d -> nat$_{64}$ -> nat$_{64}$ -> array

      $\text{\color{gray} where}$
        slice_part_sorted_spec xs l h == ... $\text{\color{gray} sort xs[l..h] up to threshold}$

    \end{lstlisting}

   \item<+-> From here on, impl-details and internal refinement steps are irrelevant

  \end{itemize}
\end{frame}
\begin{frame}[t,fragile]{Some of the Optimizations}

  \begin{itemize}
   \item<2-> unguarded insertion sort
    \begin{itemize}
     \item omit index check in insert, if $\exists$ smaller element
     \item guard controlled by flag. (\isai{insort G xs l h})
     \item specialized for \isai$G={true,false}$

    \end{itemize}
   \item<3-> move instead of swap (insert, sift-down)
    \begin{itemize}
     \item element gets overwritten in next loop iteration anyway
     \item insert: directly implemented
     \item sift-down: by refinement from version with swap

    \end{itemize}
  \end{itemize}
  \only<4->{
  \begin{itemize}
   \item manual tail-recursion optimization
    \begin{itemize}
     \item replace second \textsc{introsort\_aux} call by loop
     \item omitted in formalization
     \item but done by LLVM optimizer!
    \end{itemize}
  \end{itemize}
  }\only<1-3>{
  \begin{algorithmic}[1]
    \Procedure{insert}{$G,xs,l,i$}
      \State{$tmp\gets xs[i]$}
      \While{$({\only<2>{\color{red}}\neg G \vee l<i}) \wedge tmp < xs[i-1]$}
        \State{$\only<3>{\color{red}}xs[i]\gets xs[i-1]$}
        \State{$i\gets i-1$}
      \EndWhile
      \State{$xs[i]\gets tmp$}
    \EndProcedure
  \end{algorithmic}
  }

\end{frame}
\begin{frame}{Pdqsort: Algorithm}
  \footnotesize
  \begin{algorithmic}[1]
    \Procedure{pdqsort}{$xs,l,h$}
      \If{$h-l>1$}~\Call{pdqsort\_aux}{$\texttt{true},xs,l,h,\log(h-l)$}\EndIf \label{l:pdq:aux}
    \EndProcedure

    \Procedure{pdqsort\_aux}{$lm,xs,l,h,d$}
      \If{$h-l<\textrm{threshold}$}~\Call{insort}{$lm,xs,l,h$} \label{l:pdq:threshold}
      \Else
        \State{\Call{pivot\_to\_front}{$xs,l,h$}} \label{l:pdq:pivotfront}
        \If{$\neg lm \wedge xs[l-1] \not< xs[l]$} \label{l:pdq:eqopt}
          \State{$m\gets \Call{partition\_left}{xs,l,h}$} \label{l:pdq:partleft}
          \State{\textbf{assert} $m+1\le h$} \label{l:pdq:assert}
          \State{\Call{pdqsort\_aux}{$\false,xs,m+1,h,d$}} \label{l:pdq:recright}
        \Else
          \State{$(m,ap)\gets\Call{partition\_right}{xs,l,h}$} \label{l:pdq:part}
          \If{$m-l < \lfloor(h-l)/8\rfloor \vee h-m-1 < \lfloor(h-l)/8\rfloor$} \label{l:pdq:unbal}
            \If{${-}{-}d = 0$}~\Call{heapsort}{xs,l,h};~\Return\EndIf \label{l:pdq:heapsort}
            \State{\Call{shuffle}{xs,l,h,m}} \label{l:pdq:shuffle}
          \ElsIf{$ap \wedge \Call{maybe\_sort}{xs,l,m} \wedge \Call{maybe\_sort}{xs,m+1,h}$}\State{\Return} \label{l:pdq:maybe}
          \EndIf
          \State{\Call{pdqsort\_aux}{$lm,xs,l,m,d$}}        \label{l:pdq:rec1}
          \State{\Call{pdqsort\_aux}{$\false,xs,m+1,h,d$}}  \label{l:pdq:rec2}
        \EndIf
      \EndIf
    \EndProcedure
  \end{algorithmic}

\end{frame}
\begin{frame}{Pdqsort: Verification}
  \begin{itemize}
   \item<+-> Similar to introsort, but
    \begin{itemize}
     \item more complex
     \item different depth-limit implementation (max \#unbalanced partitions)
     \item insort inside algorithm (rather than final insort)
    \end{itemize}
   \item<+-> Verification went mostly smoothly
    \begin{itemize}
     \item heapsort, and parts of insort could be re-used
     \item had learned our lessons from introsort verification
      \begin{itemize}
       \item slightly more coarse-grained refinement steps
      \end{itemize}
     \item in-bound proofs overwhelmed Isabelle's simplifier
      \begin{itemize}
       \item solved by 'hiding' arithmetic operations behind custom constants


%
% # xxx
%   * modular approach
%   * examples from paper here
%
%   * mention pdqsort
%     * slightly coarser refinements (beware of over-engineering)
%
% # Ownership
%   * Need to sort arrays of strings (array of array of char)
%   * Challenging for separation logic based approach of Refinement Framework
%   * E.g., nth: inner array onwed by other array and result
%   * Solution: borrowing

      \end{itemize}
    \end{itemize}
  \end{itemize}
\end{frame}
\begin{frame}{Benchmarks: \only<1,3>{Introsort}\only<2,4,5>{Pdqsort} (\only<1,2,5>{64 bit integers}\only<3,4>{strings})\only<1-4>{ (Intel laptop)}\only<5>{ (AMD server)}}
  \begin{tikzpicture}
    \begin{axis}[
%       xlabel={Benchmark Set},
      xlabel near ticks,
      ylabel={Time/ms},
      legend style = {
        legend pos=outer north east,
        cells={anchor=west},
        font=\scriptsize
      },
      ybar=0pt,
%      ymin=0,
%       ymax=1e4,
      bar width=1.5pt,
      symbolic x coords={rev-sorted-end-10,rev-sorted-end-1,sorted-end-.1,almost-sorted-50,random-boolean,organ-pipe,sorted-end-10,equal,rev-sorted-middle-.1,rev-sorted,sorted-middle-1,rev-sorted-middle-10,random,almost-sorted-.1,sorted,rev-sorted-middle-1,sorted-middle-.1,almost-sorted-10,almost-sorted-1,sorted-middle-10,rev-sorted-end-.1,sorted-end-1,random-dup-10,},      xtick=data,
%       nodes near coords,
%       nodes near coords align={vertical},
      x tick label style={rotate=45,anchor=east,font=\tiny},
%       ytick={1e3,1e4},
      ymin=0,
      ymax=1.2e4,
%      restrict y to domain=0:1.1e4,
%       ytick distance = 10,
%       ytickten = {1,2,3,4,5},
    ]
      \only<1>{
      \addplot[color=red,fill=red] coordinates {
      (rev-sorted-end-10,3827)
      (rev-sorted-end-1,4186)
      (sorted-end-.1,4781)
      (almost-sorted-50,6477)
      (random-boolean,1667)
      (organ-pipe,8457)
      (sorted-end-10,6043)
      (equal,1142)
      (rev-sorted-middle-.1,3342)
      (rev-sorted,981)
      (sorted-middle-1,3799)
      (rev-sorted-middle-10,5527)
      (random,7368)
      (almost-sorted-.1,1419)
      (sorted,1438)
      (rev-sorted-middle-1,3302)
      (sorted-middle-.1,3448)
      (almost-sorted-10,3107)
      (almost-sorted-1,1590)
      (sorted-middle-10,5514)
      (rev-sorted-end-.1,6211)
      (sorted-end-1,9103)
      (random-dup-10,8175)
      };
      \addlegendentry{Isabelle-LLVM};
      \addplot[color=blue,fill=blue] coordinates {
      (rev-sorted-end-10,3787)
      (rev-sorted-end-1,3815)
      (sorted-end-.1,4982)
      (almost-sorted-50,6480)
      (random-boolean,1669)
      (organ-pipe,9203)
      (sorted-end-10,6312)
      (equal,1318)
      (rev-sorted-middle-.1,3544)
      (rev-sorted,963)
      (sorted-middle-1,3376)
      (rev-sorted-middle-10,5460)
      (random,7405)
      (almost-sorted-.1,1376)
      (sorted,1349)
      (rev-sorted-middle-1,3380)
      (sorted-middle-.1,3548)
      (almost-sorted-10,2994)
      (almost-sorted-1,1740)
      (sorted-middle-10,5574)
      (rev-sorted-end-.1,5880)
      (sorted-end-1,9601)
      (random-dup-10,8190)
      };
      \addlegendentry{libstdc++ (unverified)};
      }
      \only<2>{
        \addplot[color=red,fill=red] coordinates {
        (rev-sorted-end-10,4278)
        (rev-sorted-end-1,2245)
        (sorted-end-.1,1817)
        (almost-sorted-50,6985)
        (random-boolean,474)
        (organ-pipe,3278)
        (sorted-end-10,4008)
        (equal,135)
        (rev-sorted-middle-.1,2012)
        (rev-sorted,268)
        (sorted-middle-1,2603)
        (rev-sorted-middle-10,6006)
        (random,8066)
        (almost-sorted-.1,891)
        (sorted,200)
        (rev-sorted-middle-1,2613)
        (sorted-middle-.1,2015)
        (almost-sorted-10,2982)
        (almost-sorted-1,1204)
        (sorted-middle-10,6005)
        (rev-sorted-end-.1,1795)
        (sorted-end-1,2278)
        (random-dup-10,7767)
        };
        \addlegendentry{Isabelle-LLVM};

        \addplot[color=blue,fill=blue] coordinates {
        (rev-sorted-end-10,4072)
        (rev-sorted-end-1,2215)
        (sorted-end-.1,1816)
        (almost-sorted-50,6606)
        (random-boolean,458)
        (organ-pipe,3173)
        (sorted-end-10,3840)
        (equal,148)
        (rev-sorted-middle-.1,1992)
        (rev-sorted,230)
        (sorted-middle-1,2530)
        (rev-sorted-middle-10,5673)
        (random,7505)
        (almost-sorted-.1,849)
        (sorted,154)
        (rev-sorted-middle-1,2529)
        (sorted-middle-.1,1992)
        (almost-sorted-10,2844)
        (almost-sorted-1,1123)
        (sorted-middle-10,5673)
        (rev-sorted-end-.1,1761)
        (sorted-end-1,2260)
        (random-dup-10,7220)
        };
        \addlegendentry{Boost (unverified)};
      }
      \only<3>{
        \addplot[color=red,fill=red] coordinates {
        (rev-sorted-end-10,2664)
        (rev-sorted-end-1,2956)
        (sorted-end-.1,6756)
        (almost-sorted-50,4324)
        (random-boolean,5753)
        (organ-pipe,7127)
        (sorted-end-10,2632)
        (equal,2943)
        (rev-sorted-middle-.1,2249)
        (rev-sorted,1412)
        (sorted-middle-1,2468)
        (rev-sorted-middle-10,3002)
        (random,5147)
        (almost-sorted-.1,1687)
        (sorted,1991)
        (rev-sorted-middle-1,2264)
        (sorted-middle-.1,2537)
        (almost-sorted-10,3200)
        (almost-sorted-1,1932)
        (sorted-middle-10,3027)
        (rev-sorted-end-.1,4665)
        (sorted-end-1,3425)
        (random-dup-10,5744)
        };
        \addlegendentry{Isabelle-LLVM};
        \addplot[color=blue,fill=blue] coordinates {
        (rev-sorted-end-10,2707)
        (rev-sorted-end-1,3366)
        (sorted-end-.1,7080)
        (almost-sorted-50,4923)
        (random-boolean,5487)
        (organ-pipe,6658)
        (sorted-end-10,2655)
        (equal,2763)
        (rev-sorted-middle-.1,2171)
        (rev-sorted,1373)
        (sorted-middle-1,2336)
        (rev-sorted-middle-10,3504)
        (random,6000)
        (almost-sorted-.1,1649)
        (sorted,1949)
        (rev-sorted-middle-1,2572)
        (sorted-middle-.1,2387)
        (almost-sorted-10,2742)
        (almost-sorted-1,2166)
        (sorted-middle-10,3055)
        (rev-sorted-end-.1,4627)
        (sorted-end-1,3099)
        (random-dup-10,5843)
        };
        \addlegendentry{libstdc++ (unverified)};
      }
      \only<4>{
        \addplot[color=red,fill=red] coordinates {
        (rev-sorted-end-10,2341)
        (rev-sorted-end-1,1855)
        (sorted-end-.1,1756)
        (almost-sorted-50,3790)
        (random-boolean,609)
        (organ-pipe,2681)
        (sorted-end-10,2320)
        (equal,315)
        (rev-sorted-middle-.1,1711)
        (rev-sorted,440)
        (sorted-middle-1,2073)
        (rev-sorted-middle-10,2877)
        (random,4404)
        (almost-sorted-.1,1193)
        (sorted,315)
        (rev-sorted-middle-1,1904)
        (sorted-middle-.1,1970)
        (almost-sorted-10,2491)
        (almost-sorted-1,1507)
        (sorted-middle-10,2888)
        (rev-sorted-end-.1,1722)
        (sorted-end-1,1824)
        (random-dup-10,4731)
        };
        \addlegendentry{Isabelle-LLVM};
        \addplot[color=blue,fill=blue] coordinates {
        (rev-sorted-end-10,2359)
        (rev-sorted-end-1,1869)
        (sorted-end-.1,1804)
        (almost-sorted-50,3735)
        (random-boolean,609)
        (organ-pipe,2808)
        (sorted-end-10,2320)
        (equal,273)
        (rev-sorted-middle-.1,1694)
        (rev-sorted,443)
        (sorted-middle-1,2088)
        (rev-sorted-middle-10,2834)
        (random,4414)
        (almost-sorted-.1,1208)
        (sorted,319)
        (rev-sorted-middle-1,1912)
        (sorted-middle-.1,1995)
        (almost-sorted-10,2464)
        (almost-sorted-1,1529)
        (sorted-middle-10,2879)
        (rev-sorted-end-.1,1712)
        (sorted-end-1,1903)
        (random-dup-10,4749)
        };
        \addlegendentry{Boost (unverified)};
      }
      \only<5>{
        \addplot[color=red,fill=red] coordinates {
        (rev-sorted-end-10,7879)
        (rev-sorted-end-1,5177)
        (sorted-end-.1,4861)
        (almost-sorted-50,10071)
        (random-boolean,898)
        (organ-pipe,7513)
        (sorted-end-10,7611)
        (equal,408)
        (rev-sorted-middle-.1,4941)
        (rev-sorted,713)
        (sorted-middle-1,5522)
        (rev-sorted-middle-10,10144)
        (random,11658)
        (almost-sorted-.1,2863)
        (sorted,494)
        (rev-sorted-middle-1,5644)
        (sorted-middle-.1,4992)
        (almost-sorted-10,5512)
        (almost-sorted-1,3371)
        (sorted-middle-10,10054)
        (rev-sorted-end-.1,4765)
        (sorted-end-1,5293)
        (random-dup-10,11222)
        };
        \addlegendentry{Isabelle-LLVM};
        \addplot[color=blue,fill=blue] coordinates {
        (rev-sorted-end-10,8625)
        (rev-sorted-end-1,5773)
        (sorted-end-.1,5216)
        (almost-sorted-50,10583)
        (random-boolean,944)
        (organ-pipe,8437)
        (sorted-end-10,8085)
        (equal,424)
        (rev-sorted-middle-.1,5554)
        (rev-sorted,746)
        (sorted-middle-1,6296)
        (rev-sorted-middle-10,11009)
        (random,12487)
        (almost-sorted-.1,3165)
        (sorted,489)
        (rev-sorted-middle-1,6185)
        (sorted-middle-.1,5565)
        (almost-sorted-10,5940)
        (almost-sorted-1,3726)
        (sorted-middle-10,11001)
        (rev-sorted-end-.1,5262)
        (sorted-end-1,5930)
        (random-dup-10,11846)
        };
        \addlegendentry{Boost (unverified)};
      }
    \end{axis}
  \end{tikzpicture}

  Sorting \only<1-2,5>{$100\cdot10^6$ \texttt{uint64}s}\only<3-4>{$10\cdot10^6$ strings} on
  \only<1-4>{Intel Core i7-8665U CPU, 32GiB RAM}\only<5>{AMD Opteron 6176, 128GiB RAM}

\end{frame}
\begin{frame}{In the paper/formalization}
  \begin{itemize}
   \item<+-> Sorting of strings
    \begin{itemize}
     \item requires borrowing to access complex elements of array
    \end{itemize}
   \item<+-> Sorting with parameterized comparison functions
    \begin{itemize}
     \item E.g. $i<j$ iff $a[i]<a[j]$, for array $a$
     \item Engineering challenge
     \item Refinement: late introduction of parameter, abstract proofs unchanged
    \end{itemize}
   \item<+-> More benchmarks


  \end{itemize}
\end{frame}
\begin{frame}{Conclusions}
  \begin{itemize}
   \item Verified state-of-the-art sorting algorithms
    \begin{itemize}
     \item using Isabelle Refinement Framework with LLVM backend
     \item as fast as libstdc++/Boost implementations
     \item $\sim9000$ lines of proof text, $\sim130$ person hours
    \end{itemize}
   \item Future work
    \begin{itemize}
     \item branch aware optimization of pdqsort
     \item stable sorting (mergesort, timsort, ...)
     \item non-comparative/hybrid sorting (radix-sort, boost::spreadsort, ...)
      \medskip
     \item Verification Engineering (analogous to software engineering)
      \begin{itemize}
       \item {\color{green} correctness} + efficiency, scalability, adaptability, reusability, dev-cost, ...

      \end{itemize}
    \end{itemize}
  \end{itemize}
  \vfill

  \includegraphics[width=3cm,align=c]{isabelle-llvm.png}~~\begin{minipage}{.7\textwidth}
    \color{black}Formalization, benchmarks \& more\\
    \url{https://www21.in.tum.de/~lammich/isabelle_llvm/}
  \end{minipage}

  \vfill
  Considering a PhD in formal verification? \url{https://tinyurl.com/PhdIsabelleLLVM}

\end{frame}
\end{document}




